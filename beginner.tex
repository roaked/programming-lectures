%%%%%%%%%%%%%%%%%%%%%%%%%%%%%%%%%%%%%%%%%%%%%%%%%%%%%%%%%%%%%%%%%%%%%%%%%%%
% Beamer Presentation
% LaTeX Template
% Version 2.0 (March 8, 2022)
%
% This template originates from:
% https://www.LaTeXTemplates.com
%
% License:
% CC BY-NC-SA 4.0 (https://creativecommons.org/licenses/by-nc-sa/4.0/)
%
%%%%%%%%%%%%%%%%%%%%%%%%%%%%%%%%%%%%%%%%%%%%%%%%%%%%%%%%%%%%%%%%%%%%%%%%%%%
%%%%%%%%%%%%%%%%%%%%%%%%%%%%%%%%%%%%%%%%%%%%%%%%%%%%%%%%%%%%%%%%%%%%%%%%%%%
%
%%%%%%%%%%%%%%%%%%%%%%%%% MADE BY RICARDO CHIN %%%%%%%%%%%%%%%%%%%%%%%%%%%%
%
%%%%%%%%%%%%%%%%%%%%%%%%%%%%%%%%%%%%%%%%%%%%%%%%%%%%%%%%%%%%%%%%%%%%%%%%%%%
%%%%%%%%%%%%%%%%%%%%%%%%%%%%%%%%%%%%%%%%%%%%%%%%%%%%%%%%%%%% December 2023
%
% github.com/roaked 
% ricardochin.com
%                                                          git: MIT License
%----------------------------------------------------------------------------------------
%	PACKAGES AND OTHER DOCUMENT CONFIGURATIONS
%----------------------------------------------------------------------------------------

\documentclass[
	11pt, 
]{beamer}

\graphicspath{{Images/}{./}} 
\usepackage{graphicx}
\usepackage{booktabs} 
% Allows the use of \toprule, \midrule and \bottomrule for better rules in tables

%----------------------------------------------------------------------------------------
%	SELECT LAYOUT THEME
%----------------------------------------------------------------------------------------

\usetheme{Boadilla} %ok
%\usetheme{CambridgeUS} %red
%\usetheme{Goettingen} % ok
%\usetheme{Singapore} %good
%\usetheme{Szeged} %good

%----------------------------------------------------------------------------------------
%	SELECT COLOR THEME
%----------------------------------------------------------------------------------------


\usecolortheme{seahorse}


%----------------------------------------------------------------------------------------
%	SELECT FONT THEME & FONTS
%----------------------------------------------------------------------------------------

\usefonttheme{default}

%\usepackage{mathptmx} % Use the Times font for serif text
\usepackage{palatino} % Use the Palatino font for serif text
%\usepackage{helvet} % Use the Helvetica font for sans serif text
\usepackage[default]{opensans} % Use the Open Sans font for sans serif text
%\usepackage[default]{FiraSans} % Use the Fira Sans font for sans serif text
%\usepackage[default]{lato} % Use the Lato font for sans serif text

%----------------------------------------------------------------------------------------
%	SELECT INNER THEME
%----------------------------------------------------------------------------------------

%\useinnertheme{default}
\useinnertheme{circles}
%\useinnertheme{rectangles}
%\useinnertheme{rounded}
%\useinnertheme{inmargin}

%----------------------------------------------------------------------------------------
%	SELECT OUTER THEME
%----------------------------------------------------------------------------------------

%\useoutertheme{default}
%\useoutertheme{infolines}
%\useoutertheme{miniframes}
%\useoutertheme{smoothbars}
%\useoutertheme{sidebar}
%\useoutertheme{split}
%\useoutertheme{shadow}
%\useoutertheme{tree}
%\useoutertheme{smoothtree}

%----------------------------------------------------------------------------------
%	PRESENTATION INFORMATION
%----------------------------------------------------------------------------------

\title[Python: Basic Constructs]{Introduction to Python Programming}
\subtitle{Building the Foundation for Coding Success}
\author[Ricardo Chin]{Ricardo Chin}

\begin{document}
\begin{frame}
    \titlepage
    \begin{figure}
        \includegraphics[width=0.1\linewidth]{5848152fcef1014c0b5e4967}
    \end{figure}
    \frametitle{Workshop Details}
\end{frame}

%----------------------------------------------------------------------------------
%	TABLE OF CONTENTS SLIDE
%----------------------------------------------------------------------------------


\begin{frame}
	\frametitle{Presentation Overview}
	
	\tableofcontents
	%\tableofcontents[pausesections] % Output the table of contents (break sections up across separate slides)
\end{frame}

\section{Language Overview}
\begin{frame}{Use Cases}
\begin{block}{Simple Syntax - Beginner Friendly}
\begin{itemize}
\item Basic syntax examples (indentation for blocks, simplicity of loops, etc.)
\item Comparisons with other languages to showcase Python's readability
\end{itemize}
\end{block}
\bigskip
\begin{block}{Versatility in Different Domains}
\begin{itemize}
\item Specific use cases in scientific computing (NumPy, SciPy), web development (Django, Flask), GUI programming (Tkinter, PyQt)
\item Real-world examples of companies or projects using Python in these domains
\end{itemize}
\end{block}

\end{frame}

%%%%%%%%%%%%%%%%%%%%%%%%%%%%%%%%%%%%%%%%%%%%%%%%%%%%%%%%%%%%%%%%%%%%%%%%%%%%%%%%%%%
\begin{frame}{Use Cases}

\begin{block}{Discussion on the Standard Library and Packages}
\begin{itemize}
\item Popular third-party packages (e.g., Pandas, Matplotlib, Requests) and their significance in expanding Python's capabilities
\item Comparisons with other languages to showcase Python's readability
\end{itemize}
\end{block}
\bigskip
\begin{block}{Community and Support:}
\begin{itemize}
\item Python's has an active community, diverse user base, and strong support through forums, tutorials, and online resources.
\end{itemize}
\end{block}

\end{frame}



%%%%%%%%%%%%%%%%%%%%%%%%%%%%%%%%%%%%%%%%%%%%%%%%%%%%%%%%%%%%%%%%%%%%%%%%%%%%%%%%%%%


\begin{frame}{Python Development and Versions}
    \begin{itemize}
        \item Python 3: Current version (e.g., 3.12) released annually in October
        \bigskip
        \item Python 2: Support ended in 2019; Despite the end of support, around 10\% of developers continued to use Python 2 for various reasons, including legacy codebases and migration challenges.
    \end{itemize}

\end{frame}

%%%%%%%%%%%%%%%%%%%%%%%%%%%%%%%%%%%%%%%%%%%%%%%%%%%%%%%%%%%%%%%%%%%%%%%%%%%%%%%%%%%%%

\begin{frame}[fragile]{Four Code Examples}
    \small % Reduce font size
    \begin{minipage}[t]{0.45\textwidth}
        \begin{verbatim}
# Python snippet 1
for i in range(5):
    print(i)
        \end{verbatim}
    \end{minipage}
    \hfill
    \begin{minipage}[t]{0.45\textwidth}
        \begin{verbatim}
# Python snippet 2
x = 10
if x > 5:
    print("x is greater")
        \end{verbatim}
    \end{minipage}

    \vspace{1em} % Add vertical space between rows

    \begin{minipage}[t]{0.45\textwidth}
        \begin{verbatim}
# Python snippet 3
def greet(name):
    print("Hello, " + name)
        \end{verbatim}
    \end{minipage}
    
    \vspace{1em}
    
    \begin{minipage}[t]{0.45\textwidth}
        \begin{verbatim}
# Python snippet 4
numbers = [1, 2, 3, 4, 5]
squared = [x ** 2 
           for x in numbers]
print(squared)
        \end{verbatim}
    \end{minipage}
\end{frame}

%%%%%%%%%%%%%%%%%%%%%%%%%%%%%%%%%%%%%%%%%%%%%%%%%%%%%%%%%%%%%%%%%%%%%%%%%%%%%%%%

\begin{frame}{Installation on Windows}

    \textbf{Download from} \url{https://python.org} \smallskip
    
    \textbf{During installation:}
    \begin{itemize}
        \item Check the option "Add Python 3.x to PATH"
    \end{itemize} \smallskip

    \textbf{Verify the installation:}
    \begin{itemize}
        \item \texttt{python --version} should display the version number
        \item \texttt{pip install requests} should successfully download and install 'requests'
    \end{itemize} \smallskip

    \textbf{Installation includes:}
    \begin{itemize}
        \item Python runtime for executing Python code
        \item Interactive Python console
        \item IDLE: simple development environment
        \item PIP: package manager for installing extensions
    \end{itemize}
\end{frame}

%%%%%%%%%%%%%%%%%%%%%%%%%%%%%%%%%%%%%%%%%%%%%%%%%%%%%%%%%%%%%%%%%%%%%%%%%%%%%%%%

\begin{frame}{Interactive Python Console}

\textbf{Options for Running Python Code:}
    \begin{itemize}
        \item Writing programs as files and executing them (e.g., GUI, web apps, data processing)
        \item Typing code into an interactive console or notebook for quick calculations, experimentation, data exploration, or analysis
    \end{itemize} \smallskip


\textbf{Options Available:}
    \begin{itemize}
        \item Local installation and usage
        \item Online notebooks like Jupyter
    \end{itemize} \smallskip

\textbf{Launching Python Console:}
    \begin{itemize}
        \item Via command prompt: `python`
        \item Using the Start Menu (e.g., Python 3.12)
    \end{itemize} \smallskip
    
\textbf{Quitting:}
    \begin{itemize}
        \item Exiting with `exit()`
    \end{itemize}
\end{frame} 

%%%%%%%%%%%%%%%%%%%%%%%%%%%%%%%%%%%%%%%%%%%%%%%%%%%%%%%%%%%%%%%%%%%%%%%%%%%%%%%

\section{Variables}
\begin{frame}[fragile]{Variables}
    \textbf{Variables} \smallskip
    \begin{itemize}
        \item Examples \smallskip
            \begin{verbatim}
birth_year = 1970
current_year = 2020
age = current_year - birth_year
            \end{verbatim}\smallskip
        \item Naming conventions and rules \smallskip
            \begin{itemize}
                \item Names written in lowercase with words separated by underscores \smallskip
                \item Consist of letters, digits, and underscores only
            \end{itemize}
     \bigskip


   \item Overwriting Variables \smallskip
    \begin{verbatim}
name = "John"
name = "Jane"
a = 3
a = a + 1
    \end{verbatim}
    \end{itemize}
\end{frame}

%%%%%%%%%%%%%%%%%%%%%%%%%%%%%%%%%%%%%%%%%%%%%%%%%%%%%%%%%%%%%%%%%%%%%%%%%%%%%%%

\section{Basic Data Types and Structures}
\begin{frame}{Basic Data Types}

\textbf{Primitive Types:} \bigskip
    \begin{enumerate}
        \item \textbf{int (Integer):} Represents whole numbers, positive or negative, without any decimal points \smallskip
        \item \textbf{float (Floating-point Number):} Represents decimal numbers, including fractions and exponential values \smallskip
        \item \textbf{str (String):} Represents a sequence of characters enclosed in single or double quotes \smallskip
        \item \textbf{bool (Boolean):} Represents the truth values \texttt{True} or \texttt{False} \smallskip
        \item \textbf{None:} Denotes the absence of a value or lack of data. It's a unique type in Python \smallskip
    \end{enumerate}
\end{frame}

%%%%%%%%%%%%%%%%%%%%%%%%%%%%%%%%%%%%%%%%%%%%%%%%%%%%%%%%%%%%%%%%%%%%%%%%%%%%%%%

\begin{frame}{Basic Data Types}

\textbf{Integer (int)}

    \begin{itemize}
            \item Represents whole numbers, e.g., -5, 0, 100
            \item Supports arithmetic operations: addition, subtraction, multiplication, division, and exponentiation
            \item Operations involving integers produce integer results unless divided (/), which produces a float
    \end{itemize} \bigskip

\textbf{Floating-point Number (float)}

    \begin{itemize}
            \item Represents decimal numbers, e.g., 3.14, -0.002, 7.0
            \item Supports all arithmetic operations similar to integers
            \item Division (/) always produces a float
            \item Precision in floating-point arithmetic might lead to slight rounding errors in calculations
    \end{itemize}
    
\end{frame}

\begin{frame}[fragile]{Integer and Float Data Types}


\begin{block}{\textbf{Integer Operations}}
What happens when you perform integer operations like division or exponentiation in Python?
\end{block}

\pause
\begin{itemize}
    \item Integer division using `/` returns a float, and exponentiation uses the `**` operator
\end{itemize}
\pause

\begin{block}{\textbf{What if we'd like to round up or truncate our float value?}}
Imagine you have worked 27 overhours, and you would like to know how much that translates to available days
\end{block}
\pause



\begin{verbatim}
days = total_hours // 7.7  # Available days
# For total_hours = 27, days = 27 // 7.7 = 3 days
\end{verbatim}
\begin{verbatim}
remaining_hours = total_days % 7  # Remaining hours
# For total_days = 27, hours = 27 % 7.7 = 3.89 hours
\end{verbatim}

\end{frame}

\begin{frame}[fragile]{Integer and Float}

\begin{block}{\textbf{Integer Limits}}
 Are there any practical limitations to the size of integers in Python? How does Python handle large integers?
\end{block}
\pause
\begin{itemize}
    \item In Python, integers have arbitrary precision, meaning they can grow as large as the available memory allows without practical limitations
\end{itemize}

\begin{block}{\textbf{Scientific Notation}}
 How does Python handle very large or very small floating-point numbers?
\end{block}
\pause
\begin{itemize}
    \item Python uses scientific notation for very large or very small floating-point numbers $\rightarrow$ 1.5e6 represents 1.5 $ \cdot 10^{6}$ (1 500 000)
\end{itemize}


\end{frame}

%%%%%%%%%%%%%%%%%%%%%%%%%%%%%%%%%%%%%%%%%%%%%%%%%%%%%%%%%%%%%%%%%%%%%%%%%%%%%%%

\begin{frame}{String and Boolean}

\textbf{String (str)}
    \begin{itemize}
        \item Represents sequences of characters enclosed in single (' ') or double (" ") quotes 
        \item Supports various string manipulation operations: concatenation, slicing, formatting, etc $\rightarrow$ in built functions will be seen later
        \item Immutable - once created, a string cannot be modified in place
    \end{itemize}


\textbf{Boolean (bool)}
    \begin{itemize}
        \item Represents truth values: \texttt{True} or \texttt{False}.
        \item Essential for conditional and logical operations: if statements, while loops, etc
        \item Results of logical expressions (e.g., comparisons) are of type bool
    \end{itemize}

\end{frame}

%%%%%%%%%%%%%%%%%%%%%%%%%%%%%%%%%%%%%%%%%%%%%%%%%%%%%%%%%%%%%%%%%%%%%%%%%%%%%%%

\begin{frame}{String and Boolean}

\begin{block}{\textbf{Questions}}
\begin{enumerate}
    \item What is the difference between single, double and triple quotes?
    \item String concatenation?
    \item How can I do string indexing and slicing?
\end{enumerate}
\end{block}

\pause

\textbf{Answers}
\begin{enumerate}
    \item No significant difference between single and double quotes, except when the string itself contains one of these characters. Triple quotes are used for multiline strings
    \item String concatenation in Python involves using the + operator to combine two strings together (str1 + " " + str2)
    \item String indexing allows accessing individual characters within a string using indices. Slicing retrieves substrings by specifying a range of indices. \textbf{Indexing in Python starts at index i = 0!}
\end{enumerate}
    
\end{frame}


%%%%%%%%%%%%%%%%%%%%%%%%%%%%%%%%%%%%%%%%%%%%%%%%%%%%%%%%%%%%%%%%%%%%%%%%%%%%%%%

\begin{frame}[fragile]{String and Boolean}
\small % Reduce font size
 
\textbf{String Concatenation}   
\begin{verbatim}
s1, s2, name = "Hello", "World", Ricardo
result = s1 + " " + s2 + ". This is " + name 
\end{verbatim}

\textbf{F-strings}   
\begin{verbatim}
ans = f"{s1} {s2}. This is {name}!"
\end{verbatim}

\textbf{String Indexing and Slicing}
\hfill \textsc{$\rightarrow$ Notice the index}

\begin{verbatim}
text = "Python"
char = text[2]  # Accessing the character 't' at index 2
substring = text[1:4]  # Slicing to get 'yth'
\end{verbatim}


\begin{exampleblock}{\textbf{Tip}}
    Commonly utilized built in functions consist on slice(), strip(), reverse(), lower() and upper() for string operations
\end{exampleblock}

\end{frame}


%%%%%%%%%%%%%%%%%%%%%%%%%%%%%%%%%%%%%%%%%%%%%%%%%%%%%%%%%%%%%%%%%%%%%%%%%%%%%%%

\begin{frame}[fragile]{String and Boolean}

\textbf{How do we include characters like " in a string?}

\begin{verbatim}
string_t = "When you drink, you must say: "Prost!""
\end{verbatim}

Python treats the sequence $\backslash"$ like a single quote

\begin{verbatim}
string_t = "When you drink, you must say: \"Prost!\""
\end{verbatim}

You can also introduce line breaks using \backslash n 

\begin{verbatim}
res = "string_t\nstring_t"
print(res)
\end{verbatim}
\end{frame}

%%%%%%%%%%%%%%%%%%%%%%%%%%%%%%%%%%%%%%%%%%%%%%%%%%%%%%%%%%%%%%%%%%%%%%%%%%%%%%%

\begin{frame}[fragile]{None}
\textbf{None Data Type}
    \begin{itemize}
        \item Denotes the absence of a value or lack of data
        \item Often used to reset variables
        \item Evaluates to False in boolean contexts
    \end{itemize}

\begin{verbatim}
first_name = "Ricardo"
middle_name = None
last_name = "Chin"
\end{verbatim}
\begin{exampleblock}{\textbf{Tip}}
    Best practices include using None as a default value for optional function arguments or variables to signify missing values
\end{exampleblock}

\end{frame}

%%%%%%%%%%%%%%%%%%%%%%%%%%%%%%%%%%%%%%%%%%%%%%%%%%%%%%%%%%%%%%%%%%%%%%%%%%%%%%%

\begin{frame}{Building Data Types}


\textbf{None}
    \begin{itemize}
        \item Denotes the absence of a value or lack of data
        \item Often used to reset variables
        \item Evaluates to False in boolean contexts
    \end{itemize}
\end{frame}


\begin{alertblock}{\textbf{Operations}}
    \begin{itemize}
        \item \textbf{Type Coercion and Mixed Operations}  $\rightarrow$ When an operation involves both integers and floats, Python automatically promotes integers to floats to maintain precision
        \item \textbf{Casting between Integers and Floats} $\rightarrow$ Explicit casting can be done using int() and float() functions \smallskip

        \begin{enumerate}
            \item Integer to Float: float(5) converts the integer 5 to the float 5.0.
            \item Float to Integer: int(3.7) converts the float 3.7 to the integer 3
        \end{enumerate}
    \end{itemize}
\end{alertblock}


%%%%%%%%%%%%%%%%%%%%%%%%%%%%%%%%%%%%%%%%%%%%%%%%%%%%%%%%%%%%%%%%%%%%%%%%%%%%%%%


\section{Help and Documentation}
\begin{frame}{Help and Documentation}
    % Content for using help and documentation
\end{frame}

\section{Built-ins and the Standard Library}
\begin{frame}{Built-ins and Standard Library}
    % Content for built-in functions and standard library
\end{frame}

\section{Control Structures}
\begin{frame}{Control Structures}
    % Content for if/else and loops
\end{frame}

\subsection{if / else}
\begin{frame}{if / else}
    % Content for if / else
\end{frame}

\subsection{Loops (while, for)}
\begin{frame}{Loops}
    % Content for while and for loops
\end{frame}

\section{Functions}
\begin{frame}{Functions}
    % Content for functions
\end{frame}

\section{Code Quality and Linting}
\begin{frame}{Code Quality and Linting}
    % Content for code quality and linting
\end{frame}

\section{Debugging}
\begin{frame}{Debugging}
    % Content for debugging
\end{frame}

%%%%%%%%%%%%%%%%%%%%%%%%%%%%%%%%%%








%----------------------------------------------------------------------------------------
%	PRESENTATION BODY SLIDES
%----------------------------------------------------------------------------------------
%------------------------------------------------

% \subsection{Paragraphs and Lists}

% \begin{frame}
% 	\frametitle{Paragraphs of Text}
	
% 	Sed iaculis \alert{dapibus gravida}. Morbi sed tortor erat, nec interdum arcu. Sed id lorem lectus. Quisque viverra augue id sem ornare non aliquam nibh tristique. Aenean in ligula nisl. Nulla sed tellus ipsum. Donec vestibulum ligula non lorem vulputate fermentum accumsan neque mollis.
	
% 	\bigskip % Vertical whitespace
	
% 	% Quote example
% 	\begin{quote}
% 		Sed diam enim, sagittis nec condimentum sit amet, ullamcorper sit amet libero. Aliquam vel dui orci, a porta odio.\\
% 		--- Someone, somewhere\ldots
% 	\end{quote}
	
% 	\bigskip % Vertical whitespace
	
% 	Nullam id suscipit ipsum. Aenean lobortis commodo sem, ut commodo leo gravida vitae. Pellentesque vehicula ante iaculis arcu pretium rutrum eget sit amet purus. Integer ornare nulla quis neque ultrices lobortis.
% \end{frame}

%------------------------------------------------

%------------------------------------------------

% \subsection{Blocks}

% \begin{frame}
% 	\frametitle{Blocks of Highlighted Text}
	
% 	\begin{block}{Block Title}
% 		Lorem ipsum dolor sit amet, consectetur adipiscing elit. Integer lectus nisl, ultricies in feugiat rutrum, porttitor sit amet augue.
% 	\end{block}
	
% 	\begin{exampleblock}{Example Block Title}
% 		Aliquam ut tortor mauris. Sed volutpat ante purus, quis accumsan.
% 	\end{exampleblock}
	
% 	\begin{alertblock}{Alert Block Title}
% 		Pellentesque sed tellus purus. Class aptent taciti sociosqu ad litora torquent per conubia nostra, per inceptos himenaeos.
% 	\end{alertblock}
	
% 	\begin{block}{} % Block without title
% 		Suspendisse tincidunt sagittis gravida. Curabitur condimentum, enim sed venenatis rutrum, ipsum neque consectetur orci.
% 	\end{block}
% \end{frame}

%------------------------------------------------

% \subsection{Columns}

% \begin{frame}
% 	\frametitle{Multiple Columns}
% 	\framesubtitle{Subtitle} % Optional subtitle
	
% 	\begin{columns}[c] % The "c" option specifies centered vertical alignment while the "t" option is used for top vertical alignment
% 		\begin{column}{0.45\textwidth} % Left column width
% 			\textbf{Heading}
% 			\begin{enumerate}
% 				\item Statement
% 				\item Explanation
% 				\item Example
% 			\end{enumerate}
% 		\end{column}
% 		\begin{column}{0.5\textwidth} % Right column width
% 			Lorem ipsum dolor sit amet, consectetur adipiscing elit. Integer lectus nisl, ultricies in feugiat rutrum, porttitor sit amet augue. Aliquam ut tortor mauris. Sed volutpat ante purus, quis accumsan dolor.
% 		\end{column}
% 	\end{columns}
% \end{frame}

%------------------------------------------------

% \section{Table and Figure Examples}

% \subsection{Table}

% \begin{frame}
% 	\frametitle{Table}
% 	\framesubtitle{Subtitle} % Optional subtitle
	
% 	\begin{table}
% 		\begin{tabular}{l l l}
% 			\toprule
% 			\textbf{Treatments} & \textbf{Response 1} & \textbf{Response 2}\\
% 			\midrule
% 			Treatment 1 & 0.0003262 & 0.562 \\
% 			Treatment 2 & 0.0015681 & 0.910 \\
% 			Treatment 3 & 0.0009271 & 0.296 \\
% 			\bottomrule
% 		\end{tabular}
% 		\caption{Table caption}
% 	\end{table}
% \end{frame}

%------------------------------------------------

\end{document} 