%%%%%%%%%%%%%%%%%%%%%%%%%%%%%%%%%%%%%%%%%%%%%%%%%%%%%%%%%%%%%%%%%%%%%%%%%%%
% Beamer Presentation
% LaTeX Template
% Version 2.0 (March 8, 2022)
%
% This template originates from:
% https://www.LaTeXTemplates.com
%
% License:
% CC BY-NC-SA 4.0 (https://creativecommons.org/licenses/by-nc-sa/4.0/)
%
%%%%%%%%%%%%%%%%%%%%%%%%%%%%%%%%%%%%%%%%%%%%%%%%%%%%%%%%%%%%%%%%%%%%%%%%%%%
%%%%%%%%%%%%%%%%%%%%%%%%%%%%%%%%%%%%%%%%%%%%%%%%%%%%%%%%%%%%%%%%%%%%%%%%%%%
%
%%%%%%%%%%%%%%%%%%%%%%%%% MADE BY RICARDO CHIN %%%%%%%%%%%%%%%%%%%%%%%%%%%%
%
%%%%%%%%%%%%%%%%%%%%%%%%%%%%%%%%%%%%%%%%%%%%%%%%%%%%%%%%%%%%%%%%%%%%%%%%%%%
%%%%%%%%%%%%%%%%%%%%%%%%%%%%%%%%%%%%%%%%%%%%%%%%%%%%%%%%%%%% December 2023
%
% github.com/roaked 
% ricardochin.com
%                                                          git: MIT License
%----------------------------------------------------------------------------------------
%	PACKAGES AND OTHER DOCUMENT CONFIGURATIONS
%----------------------------------------------------------------------------------------

\documentclass[
	11pt, 
]{beamer}

\graphicspath{{Images/}{./}} 
\usepackage{graphicx}
\usepackage{booktabs} 
% Allows the use of \toprule, \midrule and \bottomrule for better rules in tables

%----------------------------------------------------------------------------------------
%	SELECT LAYOUT THEME
%----------------------------------------------------------------------------------------

\usetheme{Boadilla} %ok
%\usetheme{CambridgeUS} %red
%\usetheme{Goettingen} % ok
%\usetheme{Singapore} %good
%\usetheme{Szeged} %good

%----------------------------------------------------------------------------------------
%	SELECT COLOR THEME
%----------------------------------------------------------------------------------------


\usecolortheme{seahorse}


%----------------------------------------------------------------------------------------
%	SELECT FONT THEME & FONTS
%----------------------------------------------------------------------------------------

\usefonttheme{default}

%\usepackage{mathptmx} % Use the Times font for serif text
\usepackage{palatino} % Use the Palatino font for serif text
%\usepackage{helvet} % Use the Helvetica font for sans serif text
\usepackage[default]{opensans} % Use the Open Sans font for sans serif text
%\usepackage[default]{FiraSans} % Use the Fira Sans font for sans serif text
%\usepackage[default]{lato} % Use the Lato font for sans serif text

%----------------------------------------------------------------------------------------
%	SELECT INNER THEME
%----------------------------------------------------------------------------------------

%\useinnertheme{default}
\useinnertheme{circles}
%\useinnertheme{rectangles}
%\useinnertheme{rounded}
%\useinnertheme{inmargin}

%----------------------------------------------------------------------------------------
%	SELECT OUTER THEME
%----------------------------------------------------------------------------------------

%\useoutertheme{default}
%\useoutertheme{infolines}
%\useoutertheme{miniframes}
%\useoutertheme{smoothbars}
%\useoutertheme{sidebar}
%\useoutertheme{split}
%\useoutertheme{shadow}
%\useoutertheme{tree}
%\useoutertheme{smoothtree}

%----------------------------------------------------------------------------------
%	PRESENTATION INFORMATION
%----------------------------------------------------------------------------------

\title[Python: Early Fundamentals]{Introduction to Python Programming}
\subtitle{Building the Foundation for Coding Success}
\author[Ricardo Chin]{Ricardo Chin}

\begin{document}
\begin{frame}
    \titlepage
    \begin{figure}
        \includegraphics[width=0.1\linewidth]{5848152fcef1014c0b5e4967}
    \end{figure}
    \frametitle{Workshop Details}
\end{frame}

%----------------------------------------------------------------------------------
%	TABLE OF CONTENTS SLIDE
%----------------------------------------------------------------------------------


\begin{frame}
	\frametitle{Presentation Overview}
	
	\tableofcontents
	%\tableofcontents[pausesections] % Output the table of contents (break sections up across separate slides)
\end{frame}

\section{Language Overview}
\begin{frame}{Use Cases}
\begin{block}{Simple Syntax - Beginner Friendly}
\begin{itemize}
\item Basic syntax examples (indentation for blocks, simplicity of loops, etc.)
\item Comparisons with other languages to showcase Python's readability
\end{itemize}
\end{block}
\bigskip
\begin{block}{Versatility in Different Domains}
\begin{itemize}
\item Specific use cases in scientific computing (NumPy, SciPy), web development (Django, Flask), GUI programming (Tkinter, PyQt)
\item Real-world examples of companies or projects using Python in these domains
\end{itemize}
\end{block}

\end{frame}

%%%%%%%%%%%%%%%%%%%%%%%%%%%%%%%%%%%%%%%%%%%%%%%%%%%%%%%%%%%%%%%%%%%%%%%%%%%%%%%%%%%
\begin{frame}{Use Cases}

\begin{block}{Discussion on the Standard Library and Packages}
\begin{itemize}
\item Popular third-party packages (e.g., Pandas, Matplotlib, Requests) and their significance in expanding Python's capabilities
\item Comparisons with other languages to showcase Python's readability
\end{itemize}
\end{block}
\bigskip
\begin{block}{Community and Support:}
\begin{itemize}
\item Python's has an active community, diverse user base, and strong support through forums, tutorials, and online resources.
\end{itemize}
\end{block}

\end{frame}



%%%%%%%%%%%%%%%%%%%%%%%%%%%%%%%%%%%%%%%%%%%%%%%%%%%%%%%%%%%%%%%%%%%%%%%%%%%%%%%%%%%


\begin{frame}{Python Development and Versions}
    \begin{itemize}
        \item Python 3: Current version (e.g., 3.12) released annually in October
        \bigskip
        \item Python 2: Support ended in 2019; Despite the end of support, around 10\% of developers continued to use Python 2 for various reasons, including legacy codebases and migration challenges.
    \end{itemize}

\end{frame}

%%%%%%%%%%%%%%%%%%%%%%%%%%%%%%%%%%%%%%%%%%%%%%%%%%%%%%%%%%%%%%%%%%%%%%%%%%%%%%%%%%%%%

\begin{frame}[fragile]{Four Code Examples}
    \small % Reduce font size
    \begin{minipage}[t]{0.45\textwidth}
        \begin{verbatim}
# Python snippet 1
for i in range(5):
    print(i)
        \end{verbatim}
    \end{minipage}
    \hfill
    \begin{minipage}[t]{0.45\textwidth}
        \begin{verbatim}
# Python snippet 2
x = 10
if x > 5:
    print("x is greater")
        \end{verbatim}
    \end{minipage}

    \vspace{1em} % Add vertical space between rows

    \begin{minipage}[t]{0.45\textwidth}
        \begin{verbatim}
# Python snippet 3
def greet(name):
    print("Hello, " + name)
        \end{verbatim}
    \end{minipage}
    
    \vspace{1em}
    
    \begin{minipage}[t]{0.45\textwidth}
        \begin{verbatim}
# Python snippet 4
numbers = [1, 2, 3, 4, 5]
squared = [x ** 2 
           for x in numbers]
print(squared)
        \end{verbatim}
    \end{minipage}
\end{frame}

%%%%%%%%%%%%%%%%%%%%%%%%%%%%%%%%%%%%%%%%%%%%%%%%%%%%%%%%%%%%%%%%%%%%%%%%%%%%%%%%

\begin{frame}{Installation on Windows}

    \textbf{Download from} \url{https://python.org} \smallskip
    
    \textbf{During installation:}
    \begin{itemize}
        \item Check the option "Add Python 3.x to PATH"
    \end{itemize} \smallskip

    \textbf{Verify the installation:}
    \begin{itemize}
        \item \texttt{python --version} should display the version number
        \item \texttt{pip install requests} should successfully download and install 'requests'
    \end{itemize} \smallskip

    \textbf{Installation includes:}
    \begin{itemize}
        \item Python runtime for executing Python code
        \item Interactive Python console
        \item IDLE: simple development environment
        \item PIP: package manager for installing extensions
    \end{itemize}
\end{frame}

%%%%%%%%%%%%%%%%%%%%%%%%%%%%%%%%%%%%%%%%%%%%%%%%%%%%%%%%%%%%%%%%%%%%%%%%%%%%%%%%

\begin{frame}{Interactive Python Console}

\textbf{Options for Running Python Code:}
    \begin{itemize}
        \item Writing programs as files and executing them (e.g., GUI, web apps, data processing)
        \item Typing code into an interactive console or notebook for quick calculations, experimentation, data exploration, or analysis
    \end{itemize} \smallskip


\textbf{Options Available:}
    \begin{itemize}
        \item Local installation and usage
        \item Online notebooks like Jupyter
    \end{itemize} \smallskip

\textbf{Launching Python Console:}
    \begin{itemize}
        \item Via command prompt: `python`
        \item Using the Start Menu (e.g., Python 3.12)
    \end{itemize} \smallskip
    
\textbf{Quitting:}
    \begin{itemize}
        \item Exiting with `exit()`
    \end{itemize}
\end{frame} 

%%%%%%%%%%%%%%%%%%%%%%%%%%%%%%%%%%%%%%%%%%%%%%%%%%%%%%%%%%%%%%%%%%%%%%%%%%%%%%%


\begin{frame}{Integrated Development Environments (IDEs)}
\begin{exampleblock}{\textbf{Available Open-source IDEs}}
\begin{itemize}
    \item VS Code 

    \item PyCharm

    \item Spyder
\end{itemize}
\end{exampleblock}\bigskip

\textbf{VS Code Setup}

\begin{itemize}
    \item Installation:
    \begin{enumerate}
        \item  Open the extensions view from the left sidebar (fifth icon)
        \item Search and install the extension named "Python" by Microsoft
    \end{enumerate} \bigskip
    
    \item Registering Python Installation with VS Code:
    \begin{enumerate}
        \item Open the command palette (F1 or Ctrl + Shift + P)
        \item Search for \texttt{"Python: Select Interpreter."}
        \item Choose the desired Python version (usually only one available)
     \end{enumerate}

     
\end{itemize}

\end{frame}

%%%%%%%%%%%%%%%%%%%%%%%%%%%%%%%%%%%%%%%%%%%%%%%%%%%%%%%%%%%%%%%%%%%%%%%%%%%%%%%

\section{Variables}
\begin{frame}[fragile]{Variables}
    \begin{itemize}
        \item Naming conventions and rules \smallskip
                \begin{itemize}
                    \item Written in lowercase with words separated by underscores \smallskip
                    \item Consist of letters, digits, and underscores only
                \end{itemize} \bigskip
            
        \item Variable Examples \smallskip
            \begin{verbatim}
birth_year = 2000
current_year = 2024
age = current_year - birth_year
            \end{verbatim}\smallskip
    

   \item Overwriting Variables \smallskip
    \begin{verbatim}
birth_year = 1996
birth_year = 1997
age = 27
age = 27 + 1
    \end{verbatim}
    \item \textbf{Printing using print(\texttt{variable\_name})}
    \end{itemize}
\end{frame}

%%%%%%%%%%%%%%%%%%%%%%%%%%%%%%%%%%%%%%%%%%%%%%%%%%%%%%%%%%%%%%%%%%%%%%%%%%%%%%%

\section{Basic Data Types and Structures}
\begin{frame}{Basic Data Types}

\textbf{Primitive Types:} \bigskip
    \begin{enumerate}
        \item \textbf{int (Integer):} Represents whole numbers, positive or negative, without any decimal points \smallskip
        \item \textbf{float (Floating-point Number):} Represents decimal numbers, including fractions and exponential values \smallskip
        \item \textbf{str (String):} Represents a sequence of characters enclosed in single or double quotes \smallskip
        \item \textbf{bool (Boolean):} Represents the truth values \texttt{True} or \texttt{False} \smallskip
        \item \textbf{None:} Denotes the absence of a value or lack of data. It's a unique type in Python \smallskip
    \end{enumerate}
\end{frame}

%%%%%%%%%%%%%%%%%%%%%%%%%%%%%%%%%%%%%%%%%%%%%%%%%%%%%%%%%%%%%%%%%%%%%%%%%%%%%%%

\begin{frame}{Basic Data Types}
\textbf{Integer (int)}

    \begin{itemize}
            \item Represents whole numbers, e.g., -5, 0, 100
            \item Supports arithmetic operations: addition, subtraction, multiplication, division, and exponentiation
            \item Operations involving integers produce integer results unless divided (/), which produces a float
    \end{itemize} \bigskip

\textbf{Floating-point Number (float)}

    \begin{itemize}
            \item Represents decimal numbers, e.g., 3.14, -0.002, 7.0
            \item Supports all arithmetic operations similar to integers
            \item Division (/) always produces a float
            \item Precision in floating-point arithmetic might lead to slight rounding errors in calculations
    \end{itemize}
    
\end{frame}

\begin{frame}[fragile]{Integer and Float Data Types}


\begin{block}{\textbf{Integer Operations}}
What happens when you perform integer operations like division or exponentiation in Python?
\end{block}

\pause
\begin{itemize}
    \item Integer division using `/` returns a float, and exponentiation uses the `**` operator
\end{itemize}
\pause

\begin{block}{\textbf{What if we'd like to round up or truncate our float value?}}
Imagine you have worked 27 overhours, and you would like to know how much that translates to available days
\end{block}
\pause



\begin{verbatim}
days = total_hours // 7.7  # Available days
# For total_hours = 27, days = 27 // 7.7 = 3 days
\end{verbatim}
\begin{verbatim}
remaining_hours = total_days % 7  # Remaining hours
# For total_days = 27, hours = 27 % 7.7 = 3.89 hours
\end{verbatim}

\end{frame}

\begin{frame}[fragile]{Integer and Float}

\begin{block}{\textbf{Integer Limits}}
 Are there any practical limitations to the size of integers in Python? How does Python handle large integers?
\end{block}
\pause
\begin{itemize}
    \item In Python, integers have arbitrary precision, meaning they can grow as large as the available memory allows without practical limitations
\end{itemize}

\begin{block}{\textbf{Scientific Notation}}
 How does Python handle very large or very small floating-point numbers?
\end{block}
\pause
\begin{itemize}
    \item Python uses scientific notation for very large or very small floating-point numbers $\rightarrow$ 1.5e6 represents 1.5 $ \cdot 10^{6}$ (1 500 000)
\end{itemize}


\end{frame}

%%%%%%%%%%%%%%%%%%%%%%%%%%%%%%%%%%%%%%%%%%%%%%%%%%%%%%%%%%%%%%%%%%%%%%%%%%%%%%%

\begin{frame}{String and Boolean}
\textbf{String (str)}
    \begin{itemize}
        \item Represents sequences of characters enclosed in single (' ') or double (" ") quotes 
        \item Supports various string manipulation operations: concatenation, slicing, formatting, etc $\rightarrow$ in built functions will be seen later
        \item Immutable - once created, a string cannot be modified in place
    \end{itemize}


\textbf{Boolean (bool)}
    \begin{itemize}
        \item Represents truth values: \texttt{True} or \texttt{False}.
        \item Essential for conditional and logical operations: if statements, while loops, etc
        \item Results of logical expressions (e.g., comparisons) are of type bool
    \end{itemize}

\end{frame}

%%%%%%%%%%%%%%%%%%%%%%%%%%%%%%%%%%%%%%%%%%%%%%%%%%%%%%%%%%%%%%%%%%%%%%%%%%%%%%%

\begin{frame}{String and Boolean}

\begin{block}{\textbf{Questions}}
\begin{enumerate}
    \item What is the difference between single, double and triple quotes?
    \item String concatenation?
    \item How can I do string indexing and slicing?
\end{enumerate}
\end{block}

\pause

\textbf{Answers}
\begin{enumerate}
    \item No significant difference between single and double quotes, except when the string itself contains one of these characters. Triple quotes are used for multiline strings
    \item String concatenation in Python involves using the + operator to combine two strings together (str1 + " " + str2)
    \item String indexing allows accessing individual characters within a string using indices. Slicing retrieves substrings by specifying a range of indices. \textbf{Indexing in Python starts at index i = 0!}
\end{enumerate}
    
\end{frame}


%%%%%%%%%%%%%%%%%%%%%%%%%%%%%%%%%%%%%%%%%%%%%%%%%%%%%%%%%%%%%%%%%%%%%%%%%%%%%%%

\begin{frame}[fragile]{String and Boolean}
\small % Reduce font size
 
\textbf{String Concatenation}   
\begin{verbatim}
s1, s2, name = "Hello", "World", Ricardo
result = s1 + " " + s2 + ". This is " + name 
\end{verbatim}

\textbf{F-strings}   
\begin{verbatim}
ans = f"{s1} {s2}. This is {name}!"
\end{verbatim}

\textbf{String Indexing and Slicing}
\hfill \textsc{$\rightarrow$ Notice the index}

\begin{verbatim}
text = "Python"
char = text[2]  # Accessing the character 't' at index 2
substring = text[1:4]  # Slicing to get 'yth'
\end{verbatim}


\begin{exampleblock}{\textbf{Tip}}
    \textbf{String Operations:} slice(), strip(), reverse(), lower() and upper()
\end{exampleblock}

\end{frame}


%%%%%%%%%%%%%%%%%%%%%%%%%%%%%%%%%%%%%%%%%%%%%%%%%%%%%%%%%%%%%%%%%%%%%%%%%%%%%%%

\begin{frame}[fragile]{String and Boolean}

\textbf{How do we include characters like " in a string?}

\begin{verbatim}
string_t = "When you drink, you must say: "Prost!""
\end{verbatim}

Python treats the sequence $\backslash"$ like a single quote

\begin{verbatim}
string_t = "When you drink, you must say: \"Prost!\""
\end{verbatim}

You can also introduce line breaks using \backslash n 

\begin{verbatim}
res = "string_t\nstring_t"
print(res)
\end{verbatim}
\end{frame}

%%%%%%%%%%%%%%%%%%%%%%%%%%%%%%%%%%%%%%%%%%%%%%%%%%%%%%%%%%%%%%%%%%%%%%%%%%%%%%%

\begin{frame}[fragile]{None}
\textbf{None Data Type}
    \begin{itemize}
        \item Denotes the absence of a value or lack of data
        \item Often used to reset variables
        \item Evaluates to False in boolean contexts
    \end{itemize}

\begin{verbatim}
first_name = "Ricardo"
middle_name = None
last_name = "Chin"
\end{verbatim}
\begin{exampleblock}{\textbf{Tip}}
    Best practices include using None as a default value for optional function arguments or variables to signify missing values
\end{exampleblock}

\end{frame}

%%%%%%%%%%%%%%%%%%%%%%%%%%%%%%%%%%%%%%%%%%%%%%%%%%%%%%%%%%%%%%%%%%%%%%%%%%%%%%%

\begin{frame}[fragile]{Variable Conversions}
    

\begin{alertblock}{\textbf{Operations}}
    \begin{itemize}
        \item \textbf{Type Coercion and Mixed Operations}  $\rightarrow$ When an operation involves both integers and floats, Python automatically promotes integers to floats to maintain precision
        \item \textbf{Casting between Integers and Floats} $\rightarrow$ Explicit casting can be done using int() and float() functions \smallskip

        \begin{enumerate}
            \item Integer to Float: float(5) converts the integer 5 to the float 5.0.
            \item Float to Integer: int(3.7) converts the float 3.7 to the integer 3
        \end{enumerate}
    \end{itemize}
\end{alertblock}

\begin{verbatim}
freedom = True and some_function()

pi = 3.1415
pi_int = int(pi)
message = "Pi is approximately " + str(pi_int)
\end{verbatim}

\end{frame}
%%%%%%%%%%%%%%%%%%%%%%%%%%%%%%%%%%%%%%%%%%%%%%%%%%%%%%%%%%%%%%%%%%%%%%%%%%%%%%%


\section{Functions}
\begin{frame}[fragile]{Functions}
    Functions are like specialized tools that perform specific tasks
    
    \begin{block}{\textbf{Example of Predefined Functions}}
    \begin{itemize}
        \item \texttt{len()} - Calculates the length of a string or list
        \item \texttt{id()} - Retrieves a unique ID for an object
        \item \texttt{type()} - Identifies the type of an object
        \item \texttt{print()} - Displays information on the screen
    \end{itemize}
    \end{block}
    
    A function can be passed the so-called input parameters and produce a result (a return value)
    
    \begin{block}{\textbf{Example of Input / Output Usage}}
    \begin{itemize}
        \item \texttt{len()} - Takes a string or list as input, returns the count
        \item \texttt{print()} - Receives various types of inputs but doesn't specifically return a value
    \end{itemize}
    \end{block}

\end{frame}
%%%%%%%%%%%%%%%%%%%%%%%%%%%%%%%%%%%%%%%%%%%%%%%%%%%%%%%%%%%%%%%%%%%%%%%%%%%%%%%
\begin{frame}[fragile]{Methods}

Methods are functions that belong to specific object types, like strings (\texttt{str})

\begin{block}{\textbf{Example of String Methods}}
\begin{itemize}
    \item \texttt{first\_name.upper()} - Changes a string to uppercase
    \item \texttt{first\_name.count("a")} - Counts occurrences of a specific character
    \item \texttt{first\_name.replace("a", "@")} - Replaces characters within a string
\end{itemize}
\end{block}
\end{frame}
%%%%%%%%%%%%%%%%%%%%%%%%%%%%%%%%%%%%%%%%%%%%%%%%%%%%%%%%%%%%%%%%%%%%%%%%%%%%%%%

\section{Composite Data Types and Structures}

\begin{frame}[fragile]{Composite Data Types and Structures}

\begin{itemize}

\item Composite data structures in programming encompass various types that allow the combination of multiple elements or data types into a single unit


\item Some composite data structures include \textbf{lists, sets, tuples and dictionaries} 
\smallskip

\begin{enumerate}
    \item \textbf{Lists} represent collections of elements, allowing duplicates and modification of their content $\rightarrow$ \texttt{[1, 2, 1, 'apple', 'banana']}\smallskip
    \item \textbf{Sets} contain an unordered collection of unique elements, discarding  $\rightarrow$ duplicates \textbf{lists, sets, tuples and dictionaries} \smallskip
    \item \textbf{Tuples} resemble lists but are immutable, meaning their elements cannot be altered once defined \smallskip
    \item \textbf{Dictionaries} store key-value pairs, offering a mapping relationship between unique keys and corresponding values
\end{enumerate}

\end{itemize}
\end{frame}

%%%%%%%%%%%%%%%%%%%%%%%%%%%%%%%%%%%%%%%%%%%%%%%%%%%%%%%%%%%%%%%%%%%%%%%%%%%%%%%

\begin{frame}[fragile]{Object Mutation}
\begin{itemize}

\item  Certain objects can undergo direct mutation, such as through methods like \texttt{.append()} or \texttt{.pop()}

\item Mutable objects like lists and dicts fall under this category

\item On the flip side, several basic objects, once created, remain immutable. Yet, they can be substituted by other objects as needed. It includes integers, floats, strings, booleans and tuples
\end{itemize}


\begin{alertblock}{Summary}
    In the following slides, I will describe briefly each of the composite data structures. For more info:
    \begin{enumerate}

\item  \begin{verbatim}
help(list)
\end{verbatim}

\item  \url{https://docs.python.org/3/library/index.html} (Google: "python library")
    \end{enumerate}
\end{alertblock}
\end{frame}


%%%%%%%%%%%%%%%%%%%%%%%%%%%%%%%%%%%%%%%%%%%%%%%%%%%%%%%%%%%%%%%%%%%%%%%%%%%%%%%

\begin{frame}[fragile]{Dictionaries}
    \scriptsize
\begin{block}{\textbf{Dictionaries}}
    
Dictionaries in Python, often referred to as hash maps in other programming languages, are mutable collections that store key-value pairs. They offer a mapping structure where each key is associated with a corresponding value

\end{block}

\begin{verbatim}
person = {  "first_name": "Ricardo",
            "last_name": "Chin",
            "nationality": "Portuguese",
            "age": 27            }
\end{verbatim}

\begin{exampleblock}{\textbf{Dictionary Characteristics}}
    \begin{itemize}
        \item Key-Value Mapping: Dictionaries map unique keys to specific values. The keys within a dictionary must be unique, but the values can be duplicates
        \item Mutable and Unordered: Dictionaries are mutable, meaning their contents can be changed after creation. \underline{Any data type can be also stored}
    \end{itemize}    
\end{exampleblock}
    
\end{frame}


%%%%%%%%%%%%%%%%%%%%%%%%%%%%%%%%%%%%%%%%%%%%%%%%%%%%%%%%%%%%%%%%%%%%%%%%%%%%%%%

\begin{frame}[fragile]{Dictionaries Operations}

\begin{itemize}
    \item \textbf{Creation} \\ \\
    Dictionaries are created using curly braces {} and follow the syntax {key1: value1, key2: value2, ...}.
    
    \item \textbf{Accessing Values} \\ \\
    Values within a dictionary are accessed by their corresponding keys using square brackets ([]) or the get() method

\begin{verbatim}
print(person["first_name"])  # Output: Ricardo
print(person.get("age"))  # Output: 27
\end{verbatim}
    \\
    \item \textbf{Accessing Values} \\ \\
     Dictionaries are mutable, allowing the addition, modification, or deletion of key-value pairs

    \item \textbf{ Iterating Through a Dictionary} \\ \\
     Python provides ways to iterate through \texttt{keys}, \texttt{values}, or both simultaneously using loops or methods like \texttt{keys()}, \texttt{values()} and \texttt{items()}
\end{itemize}
\end{frame}

%%%%%%%%%%%%%%%%%%%%%%%%%%%%%%%%%%%%%%%%%%%%%%%%%%%%%%%%%%%%%%%%%%%%%%%%%%%%%%%

\begin{frame}[fragile]{Lists}
\scriptsize
    
\begin{block}{\textbf{Lists}}
    
Lists or arrays are versatile and mutable collections that store ordered sequences of elements. They are denoted by square brackets [ ] and enable flexible operations for modification, insertion, deletion, and retrieval of elements

\end{block}
\begin{verbatim}
primes = [2, 3, 5, 7, 11]           users = ["Alice", "Bob", "Charlie"]
products = [        {"name": "IPhone 12", "price": 949},
                    {"name": "Fairphone", "price": 419},
                    {"name": "Pixel 5", "price": 799}            ]  \end{verbatim}

\begin{exampleblock}{\textbf{List Characteristics}}
    \begin{itemize}
        \item Ordered Collection: Lists maintain the order of elements as they are added and allow indexing to access specific elements by their position within the list
        \item Mutable: Elements can be modified, appended, removed, or replaced after creation. Operations like append(), insert() and pop() are common. \underline{Any data type can be stored}
    \end{itemize}    
\end{exampleblock}


\end{frame}



%%%%%%%%%%%%%%%%%%%%%%%%%%%%%%%%%%%%%%%%%%%%%%%%%%%%%%%%%%%%%%%%%%%%%%%%%%%%%%%

\begin{frame}[fragile]{Lists Operations}
\scriptsize
\begin{itemize}
    \item \textbf{Accessing Elements} \\ \\ \smallskip
    \begin{enumerate}
        \item  Elements within a list are accessed using zero-based indexing \\
    
        \item  primes[0] accesses the first element, primes[-1] accesses the last element, and slicing primes[1:3] retrieves elements from index 1 up to, but not including, index 3

    \end{enumerate}
    \smallskip
    
    \item \textbf{Modifying Elements} \\ \\ \smallskip
    Elements within lists can be modified directly by assignment

    \begin{verbatim} 
primes[0] = 13  
primes[-1] = 'Ricardo'
primes[1] = primes[0]
    \end{verbatim}
     \underline{Can you guess what happens?} \smallskip

    \item \textbf{Accessing Values} \\ \\ \smallskip
     Built-in list methods like append(), extend(), insert(), remove(), and pop() to manipulate lists are common. These methods enable various operations like adding elements, removing specific elements, or altering the list's structure

\end{itemize}
\end{frame}

%%%%%%%%%%%%%%%%%%%%%%%%%%%%%%%%%%%%%%%%%%%%%%%%%%%%%%%%%%%%%%%%%%%%%%%%%%%%%%%



\begin{frame}[fragile]{Sets}
\scriptsize
    
\begin{block}{\textbf{Sets}}
    
Sets are unique, unordered collections of elements that \textbf{do not allow duplicates}. They're denoted by curly braces {} or using the set() constructor. Primarily used for membership testing, eliminating duplicates from sequences, and performing set operations like union, intersection, and difference

\end{block}
\begin{verbatim}
# Creating a set directly
my_set = {1, 2, 3, 4, 5}

# Creating a set using the set() constructor
another_set = set([3, 4, 5, 6, 7]) \end{verbatim}

\begin{exampleblock}{\textbf{Sets Characteristics}}
    \begin{itemize}
        \item \textbf{Uniqueness}: Sets contain unique elements, ensuring that each element appears only once within the set
        \item \textbf{Hash-based Storage}: Internally, sets utilize hash tables to store elements, allowing for quick membership checks and set operations
    \end{itemize}    
\end{exampleblock}


\end{frame}



%%%%%%%%%%%%%%%%%%%%%%%%%%%%%%%%%%%%%%%%%%%%%%%%%%%%%%%%%%%%%%%%%%%%%%%%%%%%%%%

\begin{frame}[fragile]{Sets Operations}
\scriptsize
\begin{itemize}
    \item \textbf{Adding Elements} \\ \\ \smallskip

    Elements can be added to a set using the add() method or by using the union ($\vert$) operator
    
\begin{verbatim} 
my_set = {1, 2, 3}
my_set.add(4)  # Adds the element 4 to my_set
\end{verbatim}

    \item \textbf{Removing Elements}
    
    Elements can be removed using the remove() or discard() methods
    
    \begin{verbatim} 
my_set.remove(2)  # Removes the element 2 from my_set
    \end{verbatim}

    \item \textbf{Sets Tasks} \\ \\ \smallskip

     Sets support various operations like union ($\vert$), intersection (\&), difference (-), and symmetric difference (\^{})
     
\begin{verbatim} 
union_set = my_set | another_set  # Union 

intersection_set = my_set & another_set  # Intersection 

difference_set = my_set - another_set  
# Elements in my_set but not in another_set

symmetric_difference_set = my_set ^ another_set  
# Elements in either my_set or another_set but not both
\end{verbatim}

\end{itemize}
\end{frame}

%%%%%%%%%%%%%%%%%%%%%%%%%%%%%%%%%%%%%%%%%%%%%%%%%%%%%%%%%%%%%%%


%%%%%%%%%%%%%%%%%%%%%%%%%%%%%%%%%%%%%%%%%%%%%%%%%%%%%%%%%%%%%%%%%%%%%%%%%%%%%%%



\begin{frame}[fragile]{Tuples}
\scriptsize
    
\begin{block}{\textbf{Tuples}}

    Tuples

\end{block}
\begin{verbatim}
# Creating a set directly
my_set = {1, 2, 3, 4, 5}

# Creating a set using the set() constructor
another_set = set([3, 4, 5, 6, 7]) \end{verbatim}

\begin{exampleblock}{\textbf{Tuples Characteristics}}
    \begin{itemize}
        \item a
        \item b
    \end{itemize}    
\end{exampleblock}


\end{frame}



%%%%%%%%%%%%%%%%%%%%%%%%%%%%%%%%%%%%%%%%%%%%%%%%%%%%%%%%%%%%%%%%%%%%%%%%%%%%%%%

\begin{frame}[fragile]{Tuples Operations}
\scriptsize
\begin{itemize}
    \item a



    
\end{itemize}
\end{frame}

%%%%%%%%%%%%%%%%%%%%%%%%%%%%%%%%%%%%%%%%%%%%%%%%%%%%%%%%%%%%%%%%%%%%%%%%%%%%%%%
\section{First Program}
\small
\begin{frame}[fragile]{First Python Program}

We'll create a file called \texttt{greeting.py}

Our program will ask the user their name and greet them

\pause


\begin{verbatim}
print("What is your name?")
name = input() # input will always return a string
\end{verbatim}

Okay..now we wrote our name, how do we display the result?

\pause
\begin{verbatim}
print("Nice to meet you, " + name)
\end{verbatim}

For increasing readability, we might want to add comments to our code.
Comments start with # and describe code functionality.
Typically placed above the code they explain.

\begin{verbatim}
# Determine the length of the name
name_length = len(name)
\end{verbatim}

To execute the program type the command line \texttt{python greeting.py} on the terminal or the green play button in VS Code
    
\end{frame}


%%%%%%%%%%%%%%%%%%%%%%%%%%%%%%%%%%%%%%%%%%%%%%%%%%%%%%%%%%%%%%%%%%%%%%%%%%%%%%%

\begin{frame}{Exercises}


\begin{itemize}
    \item Write a program which asks the user for their name. It should respond with the number of letters in the user's name

    For this purpose use the function \texttt{len(...)} to determine the length of a string

    \item Write a program called \texttt{age.py} which will ask the user for their birth year and will respond with the user's age in the year 2022

    \item Write a program called \texttt{sum.py} which will ask the user for N numbers and will do the same of numbers in the range from 1 to N

    \item Write a program called \texttt{length\_of\_concatenation.py} which will ask the user for 2 strings, concatenate them and afterwards will return the length of the concatenated strings
\end{itemize}

\end{frame}


%%%%%%%%%%%%%%%%%%%%%%%%%%%%%%%%%%%%%%%%%%%%%%%%%%%%%%%%%%%%%%%%%%%%%%%%%%%%%%%

\begin{frame}[fragile]{Libraries and More Built-in Functions}
       \small % Reduce font size
    \begin{minipage}[t]{0.7\textwidth} % Adjust width ratio
    \textbf{Standard Python Libraries}\\
    Additional modules that can be imported
    \begin{verbatim}
import random
# Return random number between 1 - 10
random_number = random.randint(1, 10)
print("Random number:", random_number)
    \end{verbatim}
\end{minipage}
\hfill % Add horizontal space between minipages
\begin{minipage}[t]{0.29\textwidth} % Adjust width ratio
    \textbf{+ Builtin Functions}
    \begin{verbatim}
print(), input()
len(), max()
min(), open()
range(), round()
sorted(), sum()
type(), read()
    \end{verbatim}
\end{minipage}

    \vspace{1em} % Add vertical space between rows

    \begin{minipage}[t]{0.35\textwidth}
    \textbf{Other Useful Libraries}
        \begin{verbatim}
random
math
datetime
os (op. system)
collections
shutil
        \end{verbatim}
    \end{minipage}
    \hfill
    \begin{minipage}[t]{0.64\textwidth}
    \textbf{/ Use Case Examples}
        \begin{verbatim}
import random
print(random.randint(1, 6))
print(random.choice(["heads", "tails"]))

file = open("message.txt", "w")
file.write("hello\n")
file.write("world\n")
file.close()

        \end{verbatim}
    \end{minipage}

\end{frame}


%%%%%%%%%%%%%%%%%%%%%%%%%%%%%%%%%%%%%%%%%%%%%%%%%%%%%%%%%%%%%%%%%%%%%%%%%%%%%%%


%%%%%%%%%%%%%%%%%%%%%%%%%%%%%%%%%%%%%%%%%%%%%%%%%%%%%%%%%%%%%%%%%%%%%%%%%%%%%%%

\section{Control Structures}
\begin{frame}{Control Structures}
    % Content for if/else and loops
\end{frame}


%%%%%%%%%%%%%%%%%%%%%%%%%%%%%%%%%%%%%%%%%%%%%%%%%%%%%%%%%%%%%%%%%%%%%%%%%%%%%%%

\section{Interesting References}
\begin{frame}[fragile]{Interesting References}
\begin{exampleblock}{\textbf{Books}}
    
\begin{itemize}


\item \href{https://automatetheboringstuff.com/}{Automate the Boring Stuff with Python by Al Sweigart}

\item \href{http://greenteapress.com/thinkpython2/thinkpython2.pdf}{Think Python, 2nd Edition by Allen B. Downey}

\item \href{https://www.py4e.com/book.php}{Python for Everybody by Dr. Charles Severance}

\end{itemize}
\end{exampleblock}

\begin{exampleblock}{\textbf{Online Courses and Tutorials}}

\begin{itemize}

\item \href{https://www.codecademy.com/learn/learn-python-3}{Codecademy - Beginner Course}

\item \href{https://learnxinyminutes.com/docs/python/}{Learn X in Y Minutes - Python}

\item \href{https://ehmatthes.github.io/pcc_2e/cheat_sheets/cheat_sheets/}{Python Cheat Sheets by Eric Matthes}
\end{itemize}
\end{exampleblock}
\end{frame}




%----------------------------------------------------------------------------------------
%	PRESENTATION BODY SLIDES
%----------------------------------------------------------------------------------------
%------------------------------------------------

% \subsection{Paragraphs and Lists}

% \begin{frame}
% 	\frametitle{Paragraphs of Text}
	
% 	Sed iaculis \alert{dapibus gravida}. Morbi sed tortor erat, nec interdum arcu. Sed id lorem lectus. Quisque viverra augue id sem ornare non aliquam nibh tristique. Aenean in ligula nisl. Nulla sed tellus ipsum. Donec vestibulum ligula non lorem vulputate fermentum accumsan neque mollis.
	
% 	\bigskip % Vertical whitespace
	
% 	% Quote example
% 	\begin{quote}
% 		Sed diam enim, sagittis nec condimentum sit amet, ullamcorper sit amet libero. Aliquam vel dui orci, a porta odio.\\
% 		--- Someone, somewhere\ldots
% 	\end{quote}
	
% 	\bigskip % Vertical whitespace
	
% 	Nullam id suscipit ipsum. Aenean lobortis commodo sem, ut commodo leo gravida vitae. Pellentesque vehicula ante iaculis arcu pretium rutrum eget sit amet purus. Integer ornare nulla quis neque ultrices lobortis.
% \end{frame}

%------------------------------------------------

%------------------------------------------------

% \subsection{Blocks}

% \begin{frame}
% 	\frametitle{Blocks of Highlighted Text}
	
% 	\begin{block}{Block Title}
% 		Lorem ipsum dolor sit amet, consectetur adipiscing elit. Integer lectus nisl, ultricies in feugiat rutrum, porttitor sit amet augue.
% 	\end{block}
	
% 	\begin{exampleblock}{Example Block Title}
% 		Aliquam ut tortor mauris. Sed volutpat ante purus, quis accumsan.
% 	\end{exampleblock}
	
% 	\begin{alertblock}{Alert Block Title}
% 		Pellentesque sed tellus purus. Class aptent taciti sociosqu ad litora torquent per conubia nostra, per inceptos himenaeos.
% 	\end{alertblock}
	
% 	\begin{block}{} % Block without title
% 		Suspendisse tincidunt sagittis gravida. Curabitur condimentum, enim sed venenatis rutrum, ipsum neque consectetur orci.
% 	\end{block}
% \end{frame}

%------------------------------------------------

% \subsection{Columns}

% \begin{frame}
% 	\frametitle{Multiple Columns}
% 	\framesubtitle{Subtitle} % Optional subtitle
	
% 	\begin{columns}[c] % The "c" option specifies centered vertical alignment while the "t" option is used for top vertical alignment
% 		\begin{column}{0.45\textwidth} % Left column width
% 			\textbf{Heading}
% 			\begin{enumerate}
% 				\item Statement
% 				\item Explanation
% 				\item Example
% 			\end{enumerate}
% 		\end{column}
% 		\begin{column}{0.5\textwidth} % Right column width
% 			Lorem ipsum dolor sit amet, consectetur adipiscing elit. Integer lectus nisl, ultricies in feugiat rutrum, porttitor sit amet augue. Aliquam ut tortor mauris. Sed volutpat ante purus, quis accumsan dolor.
% 		\end{column}
% 	\end{columns}
% \end{frame}

%------------------------------------------------

% \section{Table and Figure Examples}

% \subsection{Table}

% \begin{frame}
% 	\frametitle{Table}
% 	\framesubtitle{Subtitle} % Optional subtitle
	
% 	\begin{table}
% 		\begin{tabular}{l l l}
% 			\toprule
% 			\textbf{Treatments} & \textbf{Response 1} & \textbf{Response 2}\\
% 			\midrule
% 			Treatment 1 & 0.0003262 & 0.562 \\
% 			Treatment 2 & 0.0015681 & 0.910 \\
% 			Treatment 3 & 0.0009271 & 0.296 \\
% 			\bottomrule
% 		\end{tabular}
% 		\caption{Table caption}
% 	\end{table}
% \end{frame}

%------------------------------------------------

\end{document} 