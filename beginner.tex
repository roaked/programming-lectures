%%%%%%%%%%%%%%%%%%%%%%%%%%%%%%%%%%%%%%%%%%%%%%%%%%%%%%%%%%%%%%%%%%%%%%%%%%%
% Beamer Presentation
% LaTeX Template
% Version 2.0 (March 8, 2022)
%
% This template originates from:
% https://www.LaTeXTemplates.com
%
% License:
% CC BY-NC-SA 4.0 (https://creativecommons.org/licenses/by-nc-sa/4.0/)
%
%%%%%%%%%%%%%%%%%%%%%%%%%%%%%%%%%%%%%%%%%%%%%%%%%%%%%%%%%%%%%%%%%%%%%%%%%%%
%%%%%%%%%%%%%%%%%%%%%%%%%%%%%%%%%%%%%%%%%%%%%%%%%%%%%%%%%%%%%%%%%%%%%%%%%%%
%
%%%%%%%%%%%%%%%%%%%%%%%%% MADE BY RICARDO CHIN %%%%%%%%%%%%%%%%%%%%%%%%%%%%
%
%%%%%%%%%%%%%%%%%%%%%%%%%%%%%%%%%%%%%%%%%%%%%%%%%%%%%%%%%%%%%%%%%%%%%%%%%%%
%%%%%%%%%%%%%%%%%%%%%%%%%%%%%%%%%%%%%%%%%%%%%%%%%%%%%%%%%%%% December 2023
%
% github.com/roaked 
% ricardochin.com
%                                                          git: MIT License
%----------------------------------------------------------------------------------------
%	PACKAGES AND OTHER DOCUMENT CONFIGURATIONS
%----------------------------------------------------------------------------------------

\documentclass[
	11pt, 
]{beamer}

\graphicspath{{Images/}{./}} 
\usepackage{graphicx}
\usepackage{booktabs} 
\usepackage{tikz}
\usetikzlibrary{shapes.arrows}
% Allows the use of \toprule, \midrule and \bottomrule for better rules in tables


\definecolor{offwhite}{HTML}{F8F8F2}
\definecolor{lightgrey}{HTML}{75715E}
\definecolor{mediumgrey}{HTML}{49483E}
\definecolor{darkgrey}{HTML}{272822}
\definecolor{blue}{HTML}{900C3F}
\definecolor{arrow}{HTML}{900C3F}

\tikzset{
    myarrow/.style={
        draw,
        fill=arrow,
        single arrow,
        minimum height=3.5ex,
        single arrow head extend=1ex
    }
}
\newcommand{\arrowup}{%
\tikz [baseline=-0.5ex]{\node [myarrow,rotate=90] {};}
}
\newcommand{\arrowdown}{%
\tikz [baseline=-1ex]{\node [myarrow,rotate=-90] {};}
}

\usepackage{tcolorbox}
\tcbuselibrary{minted,breakable,xparse,skins}

\definecolor{bg}{gray}{0.95}
\DeclareTCBListing{mintedbox}{O{}m!O{}}{%
  breakable=true,
  listing engine=minted,
  listing only,
  minted language=#2,
  minted style= monokai,
  minted options={%
    linenos,
    gobble=0,
    breaklines=true,
    breakafter=,,
    fontsize=\small,
    numbersep=8pt,
    #1},
  boxsep=0pt,
  left skip=0pt,
  right skip=0pt,
  left=20pt,
  right=0pt,
  top=1pt,
  bottom=1pt,
  arc=5pt,
  leftrule=0pt,
  rightrule=1pt,
  bottomrule=1.5pt,
  toprule=1.5pt,
  colback=darkgrey,
  colframe=blue!70,
  enhanced,
  overlay={%
    \begin{tcbclipinterior}
    \fill[blue!20!white] (frame.south west) rectangle ([xshift=15pt]frame.north west);
    \end{tcbclipinterior}},
  #3}

%----------------------------------------------------------------------------------------
%	SELECT LAYOUT THEME
%----------------------------------------------------------------------------------------

\usetheme{Boadilla} %ok
%\usetheme{CambridgeUS} %red
%\usetheme{Goettingen} % ok
%\usetheme{Singapore} %good
%\usetheme{Szeged} %good

%----------------------------------------------------------------------------------------
%	SELECT COLOR THEME
%----------------------------------------------------------------------------------------


\usecolortheme{seahorse}



%----------------------------------------------------------------------------------------
%	SELECT FONT THEME & FONTS
%----------------------------------------------------------------------------------------

\usefonttheme{default}

%\usepackage{mathptmx} % Use the Times font for serif text
\usepackage{palatino} % Use the Palatino font for serif text
%\usepackage{helvet} % Use the Helvetica font for sans serif text
\usepackage[default]{opensans} % Use the Open Sans font for sans serif text
%\usepackage[default]{FiraSans} % Use the Fira Sans font for sans serif text
%\usepackage[default]{lato} % Use the Lato font for sans serif text

%----------------------------------------------------------------------------------------
%	SELECT INNER THEME
%----------------------------------------------------------------------------------------

%\useinnertheme{default}
\useinnertheme{circles}
%\useinnertheme{rectangles}
%\useinnertheme{rounded}
%\useinnertheme{inmargin}

%----------------------------------------------------------------------------------------
%	SELECT OUTER THEME
%----------------------------------------------------------------------------------------

%\useoutertheme{default}
%\useoutertheme{infolines}
%\useoutertheme{miniframes}
%\useoutertheme{smoothbars}
%\useoutertheme{sidebar}
%\useoutertheme{split}
%\useoutertheme{shadow}
%\useoutertheme{tree}
%\useoutertheme{smoothtree}

%----------------------------------------------------------------------------------
%	PRESENTATION INFORMATION
%----------------------------------------------------------------------------------

\title[Python: Early Fundamentals]{Introduction to Python Programming}
\subtitle{Building the Foundation for Coding Success}
\author[Ricardo Chin]{Ricardo Chin}

\begin{document}
\begin{frame}
    \titlepage
    \begin{figure}
        \includegraphics[width=0.1\linewidth]{5848152fcef1014c0b5e4967}
    \end{figure}
    \frametitle{Workshop Details}
\end{frame}

%----------------------------------------------------------------------------------
%	TABLE OF CONTENTS SLIDE
%----------------------------------------------------------------------------------


\begin{frame}
	\frametitle{Presentation Overview}
	
	\tableofcontents
	%\tableofcontents[pausesections] % Output the table of contents (break sections up across separate slides)
\end{frame}

\section{Language Overview}
\begin{frame}{Use Cases}
\begin{block}{Simple Syntax - Beginner Friendly}
\begin{itemize}
\item Basic syntax examples (indentation for blocks, simplicity of loops, etc.)
\item Comparisons with other languages to showcase Python's readability
\end{itemize}
\end{block}
\bigskip
\begin{block}{Versatility in Different Domains}
\begin{itemize}
\item Specific use cases in scientific computing (NumPy, SciPy), web development (Django, Flask), GUI programming (Tkinter, PyQt)
\item Real-world examples of companies or projects using Python in these domains
\end{itemize}
\end{block}

\end{frame}

%%%%%%%%%%%%%%%%%%%%%%%%%%%%%%%%%%%%%%%%%%%%%%%%%%%%%%%%%%%%%%%%%%%%%%%%%%%%%%%%%%%
\begin{frame}{Use Cases}

\begin{block}{Discussion on the Standard Library and Packages}
\begin{itemize}
\item Popular third-party packages (e.g., Pandas, Matplotlib, Requests) and their significance in expanding Python's capabilities
\item Comparisons with other languages to showcase Python's readability
\end{itemize}
\end{block}
\bigskip
\begin{block}{Community and Support:}
\begin{itemize}
\item Python's has an active community, diverse user base, and strong support through forums, tutorials, and online resources.
\end{itemize}
\end{block}

\end{frame}



%%%%%%%%%%%%%%%%%%%%%%%%%%%%%%%%%%%%%%%%%%%%%%%%%%%%%%%%%%%%%%%%%%%%%%%%%%%%%%%%%%%


\begin{frame}{Python Development and Versions}
    \begin{itemize}
        \item Python 3: Current version (e.g., 3.12) released annually in October
        \bigskip
        \item Python 2: Support ended in 2019; Despite the end of support, around 10\% of developers continued to use Python 2 for various reasons, including legacy codebases and migration challenges.
    \end{itemize}

\end{frame}

%%%%%%%%%%%%%%%%%%%%%%%%%%%%%%%%%%%%%%%%%%%%%%%%%%%%%%%%%%%%%%%%%%%%%%%%%%%%%%%%%%%%%

\begin{frame}[fragile]{Four Code Examples}
    \small % Reduce font size
    \begin{minipage}[t]{0.42\textwidth}
        \begin{mintedbox}{python}
# Python snippet 1
for i in range(5):
    print(i)
        \end{mintedbox}
    \end{minipage}
    \hfill
    \begin{minipage}[t]{0.55\textwidth}
        \begin{mintedbox}{python}
# Python snippet 2
x = 10
if x > 5:
    print("x is greater")
        \end{mintedbox}
    \end{minipage}

    \vspace{1em} % Add vertical space between rows

    \begin{minipage}[t]{1\textwidth}
        \begin{mintedbox}{python}
# Python snippet 3
def greet(name):
    print("Hello, " + name)
        \end{mintedbox}
    \end{minipage}
    
    \vspace{1em}
    
    \begin{minipage}[t]{1\textwidth}
        \begin{mintedbox}{python}
# Python snippet 4
numbers = [1, 2, 3, 4, 5]
squared = [x ** 2 for x in numbers]
print(squared)
        \end{mintedbox}
    \end{minipage}
\end{frame}

%%%%%%%%%%%%%%%%%%%%%%%%%%%%%%%%%%%%%%%%%%%%%%%%%%%%%%%%%%%%%%%%%%%%%%%%%%%%%%%%

\begin{frame}{Installation on Windows}

    \textbf{Download from} \url{https://python.org} \smallskip
    
    \textbf{During installation:}
    \begin{itemize}
        \item Check the option "Add Python 3.x to PATH"
    \end{itemize} \smallskip

    \textbf{Verify the installation using Windows+R ---  CMD:}
    \begin{itemize}
        \item \texttt{python --version} should display the version number
        \item \texttt{pip install requests} should successfully download and install 'requests'
        \item \texttt{python.exe -m pip install --upgrade pip}
    \end{itemize} \smallskip

    \textbf{Installation includes:}
    \begin{itemize}
        \item Python runtime for executing Python code
        \item IDLE: simple development environment
        \item PIP: package manager for installing extensions
    \end{itemize}
\end{frame}

%%%%%%%%%%%%%%%%%%%%%%%%%%%%%%%%%%%%%%%%%%%%%%%%%%%%%%%%%%%%%%%%%%%%%%%%%%%%%%%


\begin{frame}{Integrated Development Environments (IDEs)}
\begin{exampleblock}{\textbf{Available Open-source IDEs}}
\begin{itemize}
    \item VS Code 

    \item PyCharm

    \item Spyder
\end{itemize}
\end{exampleblock}

\textbf{VS Code Setup}

\begin{itemize}
    \item Installation:
    \begin{enumerate}
        \item  Open the extensions view from the left sidebar (fifth icon)
        \item Search and install the extension named "Python" by Microsoft
    \end{enumerate} \smallskip
    
    \item Registering Python Installation with VS Code:
    \begin{enumerate}
        \item Open the command palette (F1 or Ctrl + Shift + P)
        \item Search for \texttt{"Python: Select Interpreter."}
        \item Choose the desired Python version (usually only one available)
        \item Create a Powershell integrated terminal
        \item Install Jupyter Notebook extension
     \end{enumerate}

     
\end{itemize}

\end{frame}

%%%%%%%%%%%%%%%%%%%%%%%%%%%%%%%%%%%%%%%%%%%%%%%%%%%%%%%%%%%%%%%%%%%%%%%%%%%%%%%

\section{Variables}
\begin{frame}[fragile]{Variables}
    \begin{itemize}
        \item Naming conventions and rules \smallskip
                \begin{itemize}
                    \item Written in lowercase with words separated by underscores \smallskip
                    \item Consist of letters, digits, and underscores only
                \end{itemize} \bigskip
            
        \item Variable Examples \smallskip
            \begin{mintedbox}{python}
birth_year = 2000
current_year = 2024
age = current_year - birth_year
            \end{mintedbox}\smallskip
    

   \item Overwriting Variables \smallskip
    \begin{mintedbox}{python}
birth_year = 1996
birth_year = 1997
age = 27
age = 27 + 1
    \end{mintedbox}
    \item \textbf{Printing using print(\texttt{variable\_name})}
    \end{itemize}
\end{frame}

%%%%%%%%%%%%%%%%%%%%%%%%%%%%%%%%%%%%%%%%%%%%%%%%%%%%%%%%%%%%%%%%%%%%%%%%%%%%%%%

\section{Basic Data Types and Structures}
\begin{frame}{Basic Data Types}

\textbf{Primitive Types:} \bigskip
    \begin{enumerate}
        \item \textbf{int (Integer):} Represents whole numbers, positive or negative, without any decimal points \smallskip
        \item \textbf{float (Floating-point Number):} Represents decimal numbers, including fractions and exponential values \smallskip
        \item \textbf{str (String):} Represents a sequence of characters enclosed in single or double quotes \smallskip
        \item \textbf{bool (Boolean):} Represents the truth values \texttt{True} or \texttt{False} \smallskip
        \item \textbf{None:} Denotes the absence of a value or lack of data. It's a unique type in Python \smallskip
    \end{enumerate}
\end{frame}

%%%%%%%%%%%%%%%%%%%%%%%%%%%%%%%%%%%%%%%%%%%%%%%%%%%%%%%%%%%%%%%%%%%%%%%%%%%%%%%

\begin{frame}{Basic Data Types}
\textbf{Integer (int)}

    \begin{itemize}
            \item Represents whole numbers, e.g., -5, 0, 100
            \item Supports arithmetic operations: addition, subtraction, multiplication, division, and exponentiation
            \item Operations involving integers produce integer results unless divided (/), which produces a float
    \end{itemize} \bigskip

\textbf{Floating-point Number (float)}

    \begin{itemize}
            \item Represents decimal numbers, e.g., 3.14, -0.002, 7.0
            \item Supports all arithmetic operations similar to integers
            \item Division (/) always produces a float
            \item Precision in floating-point arithmetic might lead to slight rounding errors in calculations
    \end{itemize}
    
\end{frame}

\begin{frame}[fragile]{Integer and Float Data Types}


\begin{block}{\textbf{Integer Operations}}
How would you do integer operations with division or exponentiation? What data type do you obtain?
\end{block}

\pause
\begin{itemize}
    \item Integer division using `/` returns a float, and exponentiation uses the `**` operator
\end{itemize}
\pause

\begin{block}{\textbf{What if we'd like to round up or truncate our float value?}}
Imagine you have worked 27 overhours and you would like to know how much that translates to available days. How do you do it?
\end{block}
\pause



\begin{mintedbox}{python}
days = total_hours // 7.7  # Available days
# For total_hours = 27, days = 27 // 7.7 = 3 days
\end{mintedbox}
\begin{mintedbox}{python}
remaining_hours = total_days % 7  # Remaining hours
# For total_days = 27, hours = 27 % 7.7 = 3.89 hours
\end{mintedbox}

\end{frame}

\begin{frame}[fragile]{Integer and Float}

\begin{block}{\textbf{Integer Limits}}
 Are there any practical limitations to the size of integers in Python? How does Python handle large integers?
\end{block}
\pause
\begin{itemize}
    \item In Python, integers have arbitrary precision, meaning they can grow as large as the available memory allows without practical limitations
\end{itemize}

\begin{block}{\textbf{Scientific Notation}}
 How does Python handle very large or very small floating-point numbers?
\end{block}
\pause
\begin{itemize}
    \item Python uses scientific notation for very large or very small floating-point numbers $\rightarrow$ 1.5e6 represents 1.5 $ \cdot 10^{6}$ (1 500 000)
\end{itemize}


\end{frame}

%%%%%%%%%%%%%%%%%%%%%%%%%%%%%%%%%%%%%%%%%%%%%%%%%%%%%%%%%%%%%%%%%%%%%%%%%%%%%%%

\begin{frame}{String and Boolean}
\textbf{String (str)}
    \begin{itemize}
        \item Represents sequences of characters enclosed in single (' ') or double (" ") quotes 
        \item Supports various string manipulation operations: concatenation, slicing, formatting, etc
        \item Immutable - once created, a string cannot be modified in place
    \end{itemize}


\textbf{Boolean (bool)}
    \begin{itemize}
        \item Represents truth values: \texttt{True} or \texttt{False}.
        \item Essential for conditional and logical operations: if statements, while loops, etc
        \item Results of logical expressions (e.g., comparisons) are of type bool
    \end{itemize}

\end{frame}

%%%%%%%%%%%%%%%%%%%%%%%%%%%%%%%%%%%%%%%%%%%%%%%%%%%%%%%%%%%%%%%%%%%%%%%%%%%%%%%

\begin{frame}{String and Boolean}

\begin{block}{\textbf{Questions}}
\begin{enumerate}
    \item What is the difference between single, double and triple quotes?
    \item String concatenation?
    \item How can I do string indexing and slicing?
\end{enumerate}
\end{block}

\pause

\textbf{Answers}
\begin{enumerate}
    \item No significant difference between single and double quotes, except when the string itself contains one of these characters. Triple quotes are used for multiline strings
    \item String concatenation in Python involves using the + operator to combine two strings together (str1 + " " + str2)
    \item String indexing allows accessing individual characters within a string using indices. Slicing retrieves substrings by specifying a range of indices. \textbf{Indexing in Python starts at index i = 0!}
\end{enumerate}
    
\end{frame}


%%%%%%%%%%%%%%%%%%%%%%%%%%%%%%%%%%%%%%%%%%%%%%%%%%%%%%%%%%%%%%%%%%%%%%%%%%%%%%%

\begin{frame}[fragile]{String and Boolean}
\small % Reduce font size
 
\textbf{String Concatenation}   
\begin{mintedbox}{python}
s1, s2, name = "Hello", "World", "Ricardo"
result = s1 + " " + s2 + ". This is " + name 
\end{mintedbox}

\textbf{F-strings}   
\begin{mintedbox}{python}
ans = f"{s1} {s2}. This is {name}!"
\end{mintedbox}

\textbf{String Indexing, Slicing and Repetition}
\hfill \textsc{$\rightarrow$ Notice the index}

\begin{mintedbox}{python}
text = "Python"
char = text[2] # Accessing the character 't' at idx 2
substring = text[1:4]  # Slicing to get 'yth'
repetition = 5 * text # int can multiply string
\end{mintedbox}


\begin{exampleblock}{\textbf{Tip}}
    \textbf{String Operations:} locate(), strip(), find(), lower() and upper()
\end{exampleblock}

\end{frame}


%%%%%%%%%%%%%%%%%%%%%%%%%%%%%%%%%%%%%%%%%%%%%%%%%%%%%%%%%%%%%%%%%%%%%%%%%%%%%%%

\begin{frame}[fragile]{String and Boolean}

\textbf{How do we include characters like " in a string?}

\begin{mintedbox}{python}
string_t = "When you drink, you must say: "Prost!""
\end{mintedbox}

Python treats the sequence $\backslash"$ like a single quote

\begin{mintedbox}{python}
string_t = "When you drink, you must say: \"Prost!\""
\end{mintedbox}

You can also introduce line breaks using \backslash n 

\begin{mintedbox}{python}
res = "string_t\nstring_t"
print(res)
\end{mintedbox}
\end{frame}

%%%%%%%%%%%%%%%%%%%%%%%%%%%%%%%%%%%%%%%%%%%%%%%%%%%%%%%%%%%%%%%%%%%%%%%%%%%%%%%

\begin{frame}[fragile]{None}
\textbf{None Data Type}
    \begin{itemize}
        \item Denotes the absence of a value or lack of data
        \item Often used to reset variables
        \item Evaluates to False in boolean contexts
    \end{itemize}

\begin{mintedbox}{python}
first_name = "Ricardo"
middle_name = None
last_name = "Chin"
\end{mintedbox}
\begin{exampleblock}{\textbf{Tip}}
    Best practices include using None as a default value for optional function arguments or variables to signify missing values
\end{exampleblock}

\end{frame}

%%%%%%%%%%%%%%%%%%%%%%%%%%%%%%%%%%%%%%%%%%%%%%%%%%%%%%%%%%%%%%%%%%%%%%%%%%%%%%%

\begin{frame}[fragile]{Variable Conversions}
    

\begin{alertblock}{\textbf{Operations}}
    \begin{itemize}
        \item \textbf{Mixed Type Operations}  $\rightarrow$ When an operation involves both integers and floats, Python automatically promotes integers to floats to maintain precision
        \item \textbf{Casting between Integers and Floats} $\rightarrow$ Explicit casting can be done using int() and float() functions \smallskip

        \begin{enumerate}
            \item Integer to Float: float(5) converts the integer 5 to the float 5.0.
            \item Float to Integer: int(3.7) converts the float 3.7 to the integer 3
        \end{enumerate}
    \end{itemize}
\end{alertblock}

\begin{mintedbox}{python}
pi = 3.1415
pi_int = int(pi)
message = "Pi is approximately " + str(pi_int)
\end{mintedbox}

\end{frame}
%%%%%%%%%%%%%%%%%%%%%%%%%%%%%%%%%%%%%%%%%%%%%%%%%%%%%%%%%%%%%%%%%%%%%%%%%%%%%%%


\section{Functions}
\begin{frame}[fragile]{Functions}
    Functions are like specialized tools that perform specific tasks
    
    \begin{block}{\textbf{Example of Predefined Functions}}
    \begin{itemize}
        \item \texttt{len()} - Calculates the length of a string or list
        \item \texttt{id()} - Retrieves a unique ID for an object
        \item \texttt{type()} - Identifies the type of an object
        \item \texttt{print()} - Displays information on the screen
    \end{itemize}
    \end{block}
    
    A function can be passed the so-called input parameters and produce a result (a return value)
    
    \begin{block}{\textbf{Example of Input / Output Usage}}
    \begin{itemize}
        \item \texttt{len()} - Takes a string or list as input, returns the count
        \item \texttt{print()} - Receives various types of inputs but doesn't specifically return a value
    \end{itemize}
    \end{block}

\end{frame}
%%%%%%%%%%%%%%%%%%%%%%%%%%%%%%%%%%%%%%%%%%%%%%%%%%%%%%%%%%%%%%%%%%%%%%%%%%%%%%%
\begin{frame}[fragile]{Methods}

Methods are functions that belong to specific object types, like strings (\texttt{str})

\begin{block}{\textbf{Example of String Methods}}
\begin{itemize}
    \item \texttt{first\_name.upper()} - Changes a string to uppercase
    \item \texttt{first\_name.count("a")} - Counts occurrences of a specific character
    \item \texttt{first\_name.replace("a", "@")} - Replaces characters within a string
\end{itemize}
\end{block}
\end{frame}
%%%%%%%%%%%%%%%%%%%%%%%%%%%%%%%%%%%%%%%%%%%%%%%%%%%%%%%%%%%%%%%%%%%%%%%%%%%%%%%

\section{Composite Data Types and Structures}

\begin{frame}[fragile]{Composite Data Types and Structures}

\begin{itemize}

\item Composite data structures in programming encompass various types that allow the combination of multiple elements or data types into a single unit

\smallskip
\smallskip

\item Some composite data structures include \textbf{lists, sets, tuples and dictionaries} 
\smallskip
\bigskip
\begin{enumerate}
    \item \textbf{Lists} \smallskip
    \item \textbf{Sets} \smallskip
    \item \textbf{Tuples} \smallskip
    \item \textbf{Dictionaries} 
\end{enumerate}

\end{itemize}
\end{frame}

%%%%%%%%%%%%%%%%%%%%%%%%%%%%%%%%%%%%%%%%%%%%%%%%%%%%%%%%%%%%%%%%%%%%%%%%%%%%%%%

\begin{frame}[fragile]{Object Mutation}
\begin{itemize}

\item  Certain objects can undergo direct mutation, such as through methods like \texttt{.append()} or \texttt{.pop()}

\item Mutable objects like lists and dicts fall under this category
\end{itemize}

\bigskip

\begin{alertblock}{Summary}
    In the following slides, I will describe briefly each of the composite data structures. For more info:
    \begin{enumerate}

\item  \texttt{help(list)}

\item  \url{https://docs.python.org/3/library/index.html} (Google: "python library")
    \end{enumerate}
\end{alertblock}
\end{frame}


%%%%%%%%%%%%%%%%%%%%%%%%%%%%%%%%%%%%%%%%%%%%%%%%%%%%%%%%%%%%%%%%%%%%%%%%%%%%%%%

\begin{frame}[fragile]{Dictionaries}
    \scriptsize
\begin{block}{\textbf{Dictionaries}}
    
Dictionaries in Python, often referred to as hash maps in other programming languages, are mutable collections that store key-value pairs. They offer a mapping structure where each key is associated with a corresponding value

\end{block}

\begin{mintedbox}{python}
person = {  "first_name": "Ricardo",
            "last_name": "Chin",
            "nationality": "Portuguese",
            "age": 27            }
\end{mintedbox}

\begin{exampleblock}{\textbf{Dictionary Characteristics}}
    \begin{itemize}
        \item Key-Value Mapping: Dictionaries map unique keys to specific values. The keys within a dictionary must be unique, but the values can be duplicates
        \item Mutable and Unordered: Dictionaries are mutable, meaning their contents can be changed after creation. \underline{Any data type can be also stored}
    \end{itemize}    
\end{exampleblock}
    
\end{frame}


%%%%%%%%%%%%%%%%%%%%%%%%%%%%%%%%%%%%%%%%%%%%%%%%%%%%%%%%%%%%%%%%%%%%%%%%%%%%%%%

\begin{frame}[fragile]{Dictionaries Operations}

\begin{itemize}
    \item \textbf{Creation} \\ \\
    Dictionaries are created using curly braces {} and follow the syntax {key1: value1, key2: value2, ...}.
    
    \item \textbf{Accessing Values} \\ \\
    Values within a dictionary are accessed by their corresponding keys using square brackets ([]) or the get() method

\begin{mintedbox}{python}
print(person["first_name"])  # Output: Ricardo
print(person.get("age"))  # Output: 27
\end{mintedbox}
    \\
    \item \textbf{Flexibility} \\ \\
     Dictionaries are mutable, allowing the addition, modification, or deletion of key-value pairs

    \item \textbf{Iterating Through a Dictionary} \\ \\
     Python provides ways to iterate through \texttt{keys}, \texttt{values}, or both simultaneously using loops or methods like \texttt{keys()}, \texttt{values()} and \texttt{items()}
\end{itemize}
\end{frame}

%%%%%%%%%%%%%%%%%%%%%%%%%%%%%%%%%%%%%%%%%%%%%%%%%%%%%%%%%%%%%%%%%%%%%%%%%%%%%%%

\begin{frame}[fragile]{Lists}
\scriptsize
    
\begin{block}{\textbf{Lists}}
    
Lists or arrays are versatile and mutable collections that store ordered sequences of elements. They are denoted by square brackets [ ] and enable flexible operations for modification, insertion, deletion, and retrieval of elements

\end{block}
\begin{mintedbox}{python}
primes = [2, 3, 5, 7, 11]           
users = ["Ricardo", "Pedro", "Henrique"]
phones = [{"name": "iPhone 15 Max Pro", "price": 1699},
          {"name": "Fairphone", "price": 599},
          {"name": "Pixel 8", "price": 999}    ]  \end{mintedbox}

\begin{exampleblock}{\textbf{List Characteristics}}
    \begin{itemize}
        \item Ordered Collection: Lists maintain the order of elements as they are added and allow indexing to access specific elements by their position within the list
        \item Mutable: Elements can be modified, appended, removed, or replaced after creation. Operations like append(), insert() and pop() are common. \underline{Any data type can be stored}
    \end{itemize}    
\end{exampleblock}


\end{frame}



%%%%%%%%%%%%%%%%%%%%%%%%%%%%%%%%%%%%%%%%%%%%%%%%%%%%%%%%%%%%%%%%%%%%%%%%%%%%%%%

\begin{frame}[fragile]{Lists Operations}
\scriptsize
\begin{itemize}
    \item \textbf{Accessing Elements} \\ \\ \smallskip
    \begin{enumerate}
        \item  Elements within a list are accessed using zero-based indexing \\
    
        \item  primes[0] accesses the first element, primes[-1] accesses the last element, and slicing primes[1:3] retrieves elements from index 1 up to, but not including, index 3

    \end{enumerate}
    \smallskip
    
    \item \textbf{Modifying Elements} \\ \\ \smallskip
    Elements within lists can be modified directly by assignment

    \begin{mintedbox}{python}
primes[0] = 13  
primes[-1] = 'Ricardo'
primes[1] = primes[0]
    \end{mintedbox}
     \underline{Can you guess what happens?} \bigskip

    \item \textbf{Accessing Values} \\ \\ \smallskip
     Built-in list methods like append(), extend(), insert(), remove(), and pop() to manipulate lists are commo

\end{itemize}
\end{frame}

%%%%%%%%%%%%%%%%%%%%%%%%%%%%%%%%%%%%%%%%%%%%%%%%%%%%%%%%%%%%%%%%%%%%%%%%%%%%%%%



\begin{frame}[fragile]{Sets}
\scriptsize
    
\begin{block}{\textbf{Sets}}
    
Sets are unique, unordered collections of elements that \textbf{do not allow duplicates}. They're denoted by curly braces {} or using the set() constructor
\end{block}

\bigskip

\begin{mintedbox}{python}
# Creating a set directly
my_set = {1, 2, 3, 4, 5}

# Creating a set using the set() constructor
another_set = set([3, 4, 5, 6, 7]) 
\end{mintedbox}

\begin{exampleblock}{\textbf{Sets Characteristics}}
    \begin{itemize}
        \item \textbf{Uniqueness}: Sets contain unique elements
        \item \textbf{Hash-based Storage}: Internally, sets utilize hash tables to store elements
    \end{itemize}    
\end{exampleblock}


\end{frame}



%%%%%%%%%%%%%%%%%%%%%%%%%%%%%%%%%%%%%%%%%%%%%%%%%%%%%%%%%%%%%%%%%%%%%%%%%%%%%%%

\begin{frame}[fragile]{Sets Operations}
\scriptsize
\begin{itemize}
    \item \textbf{Adding Elements} \\ \\ \smallskip

    Elements can be added using the add() method or by using the union ($\vert$) operator
    
\begin{mintedbox}{python} 
my_set = {1, 2, 3}
my_set.add(4) # Adds the element 4 to my_set
\end{mintedbox}

    \item \textbf{Removing Elements}
    
    Elements can be removed using the remove() or discard() methods
    
    \begin{mintedbox}{python}
my_set.remove(2) # Removes element 2 from my_set
    \end{mintedbox}

    \item \textbf{Sets Tasks} \\ \\ \smallskip

     Operations like union ($\vert$), intersection (\&), difference (-), and symmetric difference (\^{})
     
\begin{mintedbox}{python}
union_set = my_set | another_set  # Union 
inter_set = my_set & another_set  # Intersection
difference_set = my_set - another_set  
# Elements in my_set but not in another_set
symmetric_difference_set = my_set ^ another_set  
# Elements in either but not both
\end{mintedbox}

\end{itemize}
\end{frame}

%%%%%%%%%%%%%%%%%%%%%%%%%%%%%%%%%%%%%%%%%%%%%%%%%%%%%%%%%%%%%%%


%%%%%%%%%%%%%%%%%%%%%%%%%%%%%%%%%%%%%%%%%%%%%%%%%%%%%%%%%%%%%%%%%%%%%%%%%%%%%%%



\begin{frame}[fragile]{Tuples}
\small
    
\begin{block}{\textbf{Tuples}}
 Tuples are ordered collections of elements, similar to lists, but they are immutable, meaning their contents cannot be modified after creation
\end{block}
\smallskip
\begin{mintedbox}{python}
# Creating a tuple
my_tuple = (1, 'apple', 3.5, (4, 5, 6))
\end{mintedbox}

\begin{exampleblock}{\textbf{Tuples Characteristics}}
    \begin{itemize}
        \item \textbf{Ordered Sequence}: Maintain order of elements, allowing access via indexing
        \item \textbf{Immutable Nature}: Elements cannot be changed, added, or removed
        \item \textbf{Flexible Data Types}: Can hold elements all types in the same structure, similar to lists
    \end{itemize}    
\end{exampleblock}

\end{frame}

%%%%%%%%%%%%%%%%%%%%%%%%%%%%%%%%%%%%%%%%%%%%%%%%%%%%%%%%%%%%%%%%%%%%%%%%%%%%%%%

\begin{frame}[fragile]{Tuples Operations}
\small
\begin{itemize}
    \item \textbf{Accessing Elements}: Elements are accessed using zero-based indexing or slicing. my\_tuple[0] accesses the first element, my\_tuple[-1] accesses the last element

    \begin{mintedbox}{python}
# Accessing elements in a tuple
print(my_tuple[1])  # Output: 'apple'
    \end{mintedbox}

    \smallskip

    \item \textbf{Tuple Methods}: Tuples have limited methods due to their immutability

    \smallskip

    \begin{mintedbox}{python}
# Tuple methods
print(my_tuple.count('apple'))  # Counts occurrences of 'apple' 
print(my_tuple.index(3.5))  # Returns the index of 3.5 
    \end{mintedbox}
    
\end{itemize}
\end{frame}

%%%%%%%%%%%%%%%%%%%%%%%%%%%%%%%%%%%%%%%%%%%%%%%%%%%%%%%%%%%%%%%%%%%%%%%%%%%%%%%
\section{First Program}
\small
\begin{frame}[fragile]{First Python Program}

We'll create a file called \texttt{greeting.py}

Our program will ask the user their name and greet them

\pause


\begin{verbatim}
print("What is your name?")
name = input() # input will always return a string
\end{verbatim}

Okay..now we wrote our name, how do we display the result?

\pause
\begin{verbatim}
print("Nice to meet you, " + name)
\end{verbatim}

For increasing readability, we might want to add comments to our code.
Comments start with \# and describe code functionality.
Typically placed above the code they explain.

\begin{verbatim}
# Determine the length of the name
name_length = len(name)
\end{verbatim}

To execute the program type the command line \texttt{python greeting.py} on the terminal or the green play button in VS Code
    
\end{frame}


%%%%%%%%%%%%%%%%%%%%%%%%%%%%%%%%%%%%%%%%%%%%%%%%%%%%%%%%%%%%%%%%%%%%%%%%%%%%%%%

\begin{frame}{Exercises}


\begin{itemize}
    \item Write a program which asks the user for their name. It should respond with the number of letters in the user's name

    \textbf{HINT:} For this purpose use the function \texttt{len(...)} to determine the length of a string

    \vspace{.5cm}

    \item Write a program called \texttt{age.py} which will ask the user for their birth year and will respond with the user's age in the year 2022

    \vspace{.5cm}

    \item Write a program called \texttt{length\_of\_concatenation.py} which will ask the user for 2 strings, concatenate them and afterwards will return the length of the concatenated strings
\end{itemize}

\end{frame}


%%%%%%%%%%%%%%%%%%%%%%%%%%%%%%%%%%%%%%%%%%%%%%%%%%%%%%%%%%%%%%%%%%%%%%%%%%%%%%%

\begin{frame}[fragile]{Libraries and More Built-in Functions}
       \small % Reduce font size
    \begin{minipage}[t]{0.7\textwidth} % Adjust width ratio
    \textbf{Standard Python Libraries}\\
    Additional modules that can be imported
    \begin{verbatim}
import random
# Return random number between 1 - 10
random_number = random.randint(1, 10)
print("Random number:", random_number)
    \end{verbatim}
\end{minipage}
\hfill % Add horizontal space between minipages
\begin{minipage}[t]{0.29\textwidth} % Adjust width ratio
    \textbf{+ Builtin Functions}
    \begin{verbatim}
print(), input()
len(), max()
min(), open()
range(), round()
sorted(), sum()
type(), read()
    \end{verbatim}
\end{minipage}

    \vspace{1em} % Add vertical space between rows

    \begin{minipage}[t]{0.35\textwidth}
    \textbf{Other Useful Libraries}
        \begin{verbatim}
random
math
datetime
os (op. system)
collections
shutil
        \end{verbatim}
    \end{minipage}
    \hfill
    \begin{minipage}[t]{0.64\textwidth}
    \textbf{/ Use Case Examples}
        \begin{verbatim}
import random
print(random.randint(1, 6))
print(random.choice(["heads", "tails"]))

file = open("message.txt", "w")
file.write("hello\n")
file.write("world\n")
file.close()

        \end{verbatim}
    \end{minipage}

\end{frame}


%%%%%%%%%%%%%%%%%%%%%%%%%%%%%%%%%%%%%%%%%%%%%%%%%%%%%%%%%%%%%%%%%%%%%%%%%%%%%%%
%%%%%%%%%%%%%%%%%%%%%%%%%%%%%%%%%%%%%%%%%%%%%%%%%%%%%%%%%%%%%%%%%%%%%%%%%%%%%%%

\section{Control Structures}
\begin{frame}{Control Structures}
Control structures are essential components in programming languages that enable the control flow of a program

\begin{block}{\textbf{Control Structures}}
\begin{itemize}
\item if / else statements
\item loops
\begin{itemize}
    \item while loop
    \item do while loop
    \item for loop (counting loop)
    \item foreach loop
\end{itemize}
\end{itemize}

\end{block}

The structure is usually pretty straightforward

\begin{itemize}
\item if ... else ...
\item loops
\begin{itemize}
    \item while
    \item for ... in ...
    \item for ... in range(...)
\end{itemize}
\end{itemize}
\end{frame}

%%%%%%%%%%%%%%%%%%%%%%%%%%%%%%%%%%%%%%%%%%%%%%%%%%%%%%%%%%%%%%%%%%%%%%%%%%%%%%%

\begin{frame}[fragile]{Comparisons}
To utilize \texttt{if} and \texttt{while} statements, comparisons between values are necessary

\begin{verbatim}
    a = 1
    b = 0
    
    print(a == b)  # Is a equal to b?
    print(a != b)  # Is a not equal to b?
    print(a < b)   # Is a less than b?
    print(a > b)   # Is a greater than b?
    print(a <= b)  # Is a less than or equal to b?
    print(a >= b)  # Is a greater than or equal to b?

\end{verbatim}

Comparisons yield boolean values \texttt{True / False}. We can capture this outcome in a variable

\begin{verbatim}
    # Checking if 'age' is 18 or older
    is_adult = age >= 18  
\end{verbatim}


\end{frame}

%%%%%%%%%%%%%%%%%%%%%%%%%%%%%%%%%%%%%%%%%%%%%%%%%%%%%%%%%%%%%%%%%%%%%%%%%%%%%%%

\begin{frame}[fragile]{Comparisons}
It is possible to combine comparisons using \texttt{and, or} and \texttt{not}

\begin{verbatim}
if age >= 18 and country == "Austria":
    print("May drink alcohol.")

if temperature < -15 or temperature > 40:
    print("Extreme weather!")

if shopping_basket not None:
    print("There are still items left.")

name = "Alice"
other_name = "Bob

print(name == other_name) # Are names identical?
print("Alli" in name)     # Does the text contain "Alli"?
print("Ali" not in name)  # Does the text contain "Ali"?
\end{verbatim}

\end{frame}

%%%%%%%%%%%%%%%%%%%%%%%%%%%%%%%%%%%%%%%%%%%%%%%%%%%%%%%%%%%%%%%%%%%%%%%%%%%%%%%

\begin{frame}[fragile]{If Statements}
\texttt{if statements} help computers decide what to do based on certain conditions. It's like giving the computer a set of instructions

\begin{verbatim}
number = 27 

if number % 2 == 0:
    print("The number is even.")
else:
    print("The number is odd.")

\end{verbatim}

\begin{exampleblock}{\textbf{How does it work?}}
\begin{itemize}

    \item Checks if number is even by using the modulus operator \texttt{\%}
    \item If the remainder of \texttt{number / 2} is \texttt{0}, it prints "The number is even."
    \item If not, it prints "The number is odd."
\end{itemize}
\end{exampleblock}

\end{frame}

%%%%%%%%%%%%%%%%%%%%%%%%%%%%%%%%%%%%%%%%%%%%%%%%%%%%%%%%%%%%%%%%%%%%%%%%%%%%%%%

\begin{frame}[fragile]{If, elif and else Statements}
For multiple conditions to check: \texttt{if, elif and else} are used. The computer goes through each condition, executing the first one that's true

\begin{verbatim}
if number < 0:
    print("The number is negative.")
elif number == 0:
    print("The number is zero.")
elif number % 2 == 0:
    print("The number is divisible by 2.")
else:
    print("The number doesn't fall into any category.")
\end{verbatim}

\begin{block}{\textbf{Code Blocks}}
\begin{itemize}
    \item Code block is a group of statements that are executed together when a condition (like an \texttt{if statement}) is true
    \item The colon \texttt{:} signifies the start of a code block, and the indented lines following the colon are considered part of that block
\end{itemize}
\end{block}

\end{frame}

%%%%%%%%%%%%%%%%%%%%%%%%%%%%%%%%%%%%%%%%%%%%%%%%%%%%%%%%%%%%%%%%%%%%%%%%%%%%%%%

\begin{frame}[fragile]{If, elif and else Statements}

\begin{alertblock}{\textbf{Exercise: Coin Flip}}
Simulate coin flipping via \texttt{random.choice(["heads", "tails"])}

Let the user guess the outcome and tell them if they were right
\end{alertblock}

\pause

\begin{verbatim}

import random

# Simulating coin flipping and user guess
coin_outcome = random.choice(["heads", "tails"])
user_guess = input("Guess the outcome (heads or tails): ")

# Checking if the user's guess matches the outcome
if user_guess.lower() == coin_outcome:
    print("Congratulations! Your guess was correct.")
else:
    print(f"Sorry, the coin landed on {coin_outcome}.")
    print("Your guess was incorrect.")


\end{verbatim}
\end{frame}

%%%%%%%%%%%%%%%%%%%%%%%%%%%%%%%%%%%%%%%%%%%%%%%%%%%%%%%%%%%%%%%%%%%%%%%%%%%%%%%

\begin{frame}[fragile]{While Loops}

\texttt{while} loops repeatedly execute a block of code as long as a specified condition is true. They are useful when the number of iterations is not fixed beforehand and depends on certain conditions

\begin{verbatim}
    while condition:
    # code block to execute as long as it stays true
    # the condition is re-evaluated after each iteration
\end{verbatim}

\begin{block}{\textbf{Key Points}}
    \begin{enumerate}
        \item \texttt{Condition}: Repeats continuously while statement is \texttt{True}

        \item \texttt{Iteration}: The code inside the loop runs repeatedly until the condition becomes False. If the condition is initially False, the code inside the loop will not execute

        \item  \texttt{Updating Condition}: It's crucial to have a way to change the condition inside the loop to avoid an infinite loop. Usually, a variable involved in the condition is modified within the loop
    \end{enumerate}
\end{block}

\end{frame}

%%%%%%%%%%%%%%%%%%%%%%%%%%%%%%%%%%%%%%%%%%%%%%%%%%%%%%%%%%%%%%%%%%%%%%%%%%%%%%%

\begin{frame}[fragile]{While Loops}

Let's print numbers from 1 to 5 using a \texttt{while} loop:

\begin{verbatim}
num = 1
while num <= 5:
    print(num)
    
    # Incrementing variable `num` 
    num += 1 
    # Break out of the loop when num > 5 (condition)
\end{verbatim}

\begin{block}{\textbf{Advantages}}
    \begin{itemize}
        \item \texttt{Dynamic Control}: Useful when the number of iterations isn't predetermined and depends on conditions

        \item \texttt{Flexibility}: Allows for a wide range of applications such as user input, iteration based on changing conditions, and more

    \end{itemize}
\end{block}

 If the condition never becomes \texttt{False} or if there's no logic to change the condition, it causes an infinite loop

\end{frame}

%%%%%%%%%%%%%%%%%%%%%%%%%%%%%%%%%%%%%%%%%%%%%%%%%%%%%%%%%%%%%%%%%%%%%%%%%%%%%%%

\begin{frame}[fragile]{While Loops}

\begin{alertblock}{\textbf{Exercises}}
\begin{itemize}
    \item Exercise 1: Do a countdown from 10 to 0
    \item Exercise 2: Create a simple guessing game
    \item Exercise 3: Find the highest number in a list
    \item Exercise 4: Count and print the number of occurrences of a specific letter ('a') in a string \texttt{"apple and banana are fruits"}
    \item Exercise 5: Create an infinite loop of "Hello world!" prints
    \item Exercise 6: Create a team list program
    \begin{center}
    \begin{verbatim}
        Register a person or "X" to quit: 
        Ricardo
        
        Your team consists of:
        ["Anand", "Ricardo", "Simon"]
    \end{verbatim}
    \end{center}
\end{itemize}    
\end{alertblock}
\end{frame}

%%%%%%%%%%%%%%%%%%%%%%%%%%%%%%%%%%%%%%%%%%%%%%%%%%%%%%%%%%%%%%%%%%%%%%%%%%%%%%%

\begin{frame}[fragile]{For Loops}

\texttt{For} loop is like a helpful assistant that goes through a list of items, one by one, and performs a set of instructions for each item

\begin{verbatim}

names = ["Alice", "Bob", "Ricardo"]

for name in names:
    print("Hello, " + name + "!")

\end{verbatim}

\begin{exampleblock}{\textbf{How does it work?}}
    \begin{itemize}
        \item The \texttt{for} loop starts by saying, "Hey Python, let's go through each name in the \texttt{names} list."
        \item For each name, the loop performs the action inside its block of code. In this case, it says hello to each person by printing "Hello, [name]!"
    \end{itemize}
\end{exampleblock}

\end{frame}




%%%%%%%%%%%%%%%%%%%%%%%%%%%%%%%%%%%%%%%%%%%%%%%%%%%%%%%%%%%%%%%%%%%%%%%%%%%%%%%

\begin{frame}[fragile]{For Loops}

\begin{alertblock}{\textbf{Exercises}}
Assume the list = [1, 2, 3, 4, 5]
\begin{itemize}
    \item Exercise 1: Prints numbers of the list in separate lines
    \item Exercise 2: Calculate the cumulative sum of the list
    \item Exercise 3: Print only the odd numbers
    \item Exercise 4: Print a pattern using loops
        \begin{verbatim}
            *
            **
            ***
            ****
            *****
        \end{verbatim}
    \item Exercise 5: Create a list with the square valued of each number
\end{itemize}    
\end{alertblock}

\end{frame}

%%%%%%%%%%%%%%%%%%%%%%%%%%%%%%%%%%%%%%%%%%%%%%%%%%%%%%%%%%%%%%%%%%%%%%%%%%%%%%%

\begin{frame}[fragile]{Counting Loops}

Counting loops are used to iterate through a sequence of numbers or perform an action a certain number of times. In Python, \texttt{for} loops and the \texttt{range()} function are commonly used for counting loops

\begin{verbatim}
    # Counting from 0 to 9 using a for loop
    for i in range(10): # 10 excluded
        print(i)
\end{verbatim}

\begin{exampleblock}{\textbf{How does it work?}}
\begin{itemize}
    \item \texttt{range(10)} generates a sequence of numbers from 0 to 9 ([0, 1, 2, 3, 4, 5, 6, 7, 8, 9])
    \item The \texttt{for} loop iterates through each number in this sequence, storing the value in \texttt{i}, and executes the code block (in this case, printing \texttt{i})
\end{itemize}
\end{exampleblock}

\vspace{.1cm}
For counting loops, \texttt{i} is commonly used as a variable name, especially when the purpose is straightforward iteration

\end{frame}

%%%%%%%%%%%%%%%%%%%%%%%%%%%%%%%%%%%%%%%%%%%%%%%%%%%%%%%%%%%%%%%%%%%%%%%%%%%%%%%

\begin{frame}[fragile]{Counting Loops}

\begin{alertblock}{\textbf{Exercises}}
\begin{itemize}
    \item Exercise 1: Return numbers from 0 to 9 in separate lines
    \item Exercise 2: Calculate the cumulative sum of numbers from 0 to 9
    \item Exercise 3: Create multiplication table of 5
    \item Exercise 4: Return the following pattern using loops
        \begin{verbatim}
            1
            12
            123
            1234
            12345\end{verbatim}
    \item Exercise 5: Print the duplicates of a given list [3, 4, 3, 5, 6, 7, 8, 5]
    \begin{itemize}
        \item Exercise 5.1: Return the unique values of the list
        \item Exercise 5.2: Finding common elements between two lists
    \end{itemize}
\end{itemize}    
\end{alertblock}
\end{frame}

%%%%%%%%%%%%%%%%%%%%%%%%%%%%%%%%%%%%%%%%%%%%%%%%%%%%%%%%%%%%%%%%%%%%%%%%%%%%%%%

\begin{frame}[fragile]{Counting Loops}
\small
\begin{alertblock}{\textbf{Exercises}}
\begin{itemize}
    \item Exercise 6: Counting vowels in a given input string
    \item Exercise 7: Finding the longest increasing subarray of a list
    \item Exercise 8: Move zeroes to the end L = [0, 1, 0, 3, 0, 12, 5]
    \item Exercise 9: Count occurrences of each char in a string
    \item Exercise 10: Every element appears twice except one. Return it
    \item Exercise 11: Return sum(max\_value\_row) - sum(min\_value\_row) 
    \begin{verbatim}
                       [1, 3, 2]
                       [3, 5, 6]
                       [0, 0, 3]\end{verbatim} 
    \item Exercise 12: Return sum(primary\_diag) - sum(secondary\_diag)
    \item Exercise 13:  \href{https://leetcode.com/problems/two-sum/}{\texttt{LeetCode 1} --- \texttt{TwoSum}}
                       
\end{itemize}    
\end{alertblock}
\end{frame}



%%%%%%%%%%%%%%%%%%%%%%%%%%%%%%%%%%%%%%%%%%%%%%%%%%%%%%%%%%%%%%%%%%%%%%%%%%%%%%%

\begin{frame}[fragile]{Break Statement}

The \texttt{break} statement is a control flow statement used to terminate the execution of a loop prematurely

\begin{verbatim}
    while True:
    if condition:
        break  # Terminate loop if condition is met
    # Other statements
\end{verbatim}

When encountered within a loop (such as \texttt{for} or \texttt{while}), the \texttt{break} statement causes the loop to immediately stop iterating

\begin{verbatim}
    while True: # cause infinite loop
        guess = int(input("Enter a number (0 to exit): "))
        if guess == 0:
            print("Exiting the loop.")
            break
        else:
            print(f"You entered: {guess}")
\end{verbatim}

\texttt{break} is particularly useful when the loop needs to end before its natural completion based on a specific condition

\end{frame}

%%%%%%%%%%%%%%%%%%%%%%%%%%%%%%%%%%%%%%%%%%%%%%%%%%%%%%%%%%%%%%%%%%%%%%%%%%%%%%%

\section{Program Structure}
\begin{frame}[fragile]{Pass Statement}


\texttt{Pass} statement is used as a placeholder for an empty code block where Python syntax requires a statement but no action is needed

\begin{center}
    \arrowdown
\end{center}

\begin{block}{\textbf{Empty Codeblocks}}
\begin{verbatim}
# TODO: warn the user if path doesn't exist
if not os.path.exists(my_path):
    pass
\end{verbatim}
\end{block}

\texttt{if condition: pass} indicates to "do nothing" when a certain condition is met

\end{frame}

%%%%%%%%%%%%%%%%%%%%%%%%%%%%%%%%%%%%%%%%%%%%%%%%%%%%%%%%%%%%%%%%%%%%%%%%%%%%%%%

\begin{frame}[fragile]{Statements Across Lines}

Statements can span multiple lines by enclosing them in parentheses \texttt{()} without the need for line continuation characters.

\begin{center}
    \arrowdown
\end{center}

\begin{block}{\textbf{Long Assignments}}
\texttt{a = (2 + 3 + 4 + 5 + 6 + 7 + 8 + 9 + 10)} can be written across multiple lines for readability
\end{block}

\begin{verbatim}
                a = (2 + 3 + 4 + 5 + 6 +
                     7 + 8 + 9 + 10)
\end{verbatim}

\centering
\texttt{()} or alternatively using newline \texttt{\backslash}

\begin{verbatim}
                a = 2 + 3 + 4 + 5 + 6 + \
                    7 + 8 + 9 + 10
\end{verbatim}
\end{frame}

%%%%%%%%%%%%%%%%%%%%%%%%%%%%%%%%%%%%%%%%%%%%%%%%%%%%%%%%%%%%%%%%%%%%%%%%%%%%%%%

\begin{frame}[fragile]{Positional and Keyword Parameters}

\begin{enumerate}
    \item \textbf{Positional Parameters}: Passed based on their position in the function call

    \vspace{.2cm}

    \textbf{Example}: \texttt{open("this\_file.txt", "w", -1, "utf-8")}, where \texttt{"this\_file.txt"} is the file name, \texttt{"w"} is the mode, \texttt{"-1"} might be the buffer size, and \texttt{"utf-8"} is the encoding

    \vspace{.5cm}

    \item \textbf{Keyword Parameters}: Passed with a keyword and value, allowing flexibility and clarity

    \vspace{.2cm}
     
    \textbf{Example}: \texttt{open("this\_file.txt", encoding="utf-8", mode="w")}, explicitly defining the encoding and mode

    \vspace{.5cm}


    \item \textbf{Optional Parameters}:  Parameters that don’t have to be provided in a function call
    \vspace{.2cm}

    \textbf{Example}: In the previous example, parameters \texttt{encoding} and \texttt{mode}
    
\end{enumerate}

\end{frame}

%%%%%%%%%%%%%%%%%%%%%%%%%%%%%%%%%%%%%%%%%%%%%%%%%%%%%%%%%%%%%%%%%%%%%%%%%%%%%%%


\begin{frame}[fragile]{Defining Functions}

\textbf{Syntax} --- \textsc{def function\_name(parameters)}

\begin{verbatim}
    def average(a, b):
    m = (a + b) / 2
    return m
\end{verbatim}

\textbf{Usage} --- Functions encapsulate reusable blocks of code, performing specific tasks

\begin{alertblock}{Exercises}
    \begin{itemize}   
        \item \textbf{Calculating Area}: Create a function to calculate the area of a rectangle or circle
        \item \textbf{Converter}: Develop a function that converts temperature between Fahrenheit and Celsius
        \item \textbf{String Manipulation}: Write a function that reverses a given string
        \item \textbf{List Operations}: Create a function to find the maximum or minimum number in a list    
    \end{itemize} 
\end{alertblock}

\end{frame}

%%%%%%%%%%%%%%%%%%%%%%%%%%%%%%%%%%%%%%%%%%%%%%%%%%%%%%%%%%%%%%%%%%%%%%%%%%%%%%%


\begin{frame}[fragile]{Variable Scope}

\textbf{Function Scope} --- Variables defined within a function have local scope, separate from variables defined outside the function

\begin{verbatim}
m = "Hello, world"

def average(a, b):
    m = (a + b) / 2
    return m

x = average(1, 2)
print(m)  # Prints "Hello, world" as the function creates a local 'm'

\end{verbatim}

\textbf{Read Access} --- Functions can access variables from the outer scope, but modifications create local variables

\vspace{.5cm}

\textbf{Immutable vs. Mutable} --- Immutable types (like strings) cannot be modified in place; assigning creates a new local variable

\end{frame}

%%%%%%%%%%%%%%%%%%%%%%%%%%%%%%%%%%%%%%%%%%%%%%%%%%%%%%%%%%%%%%%%%%%%%%%%%%%%%%%


\begin{frame}[fragile]{Function Exercises}

\begin{alertblock}{Function Exercises}
    \begin{enumerate}
        \item Leap Year Check: Verify if a given year is a leap year based on certain criteria
        \item Prime Number Check: Determine whether a number is a prime number or not
        \item Finding Prime Numbers: Create a function that returns all prime numbers within a specified interval
        \item Fibonacci Number Calculation: Compute Fibonacci numbers within a given range
        \item Index of the Sum of Two Numbers to Target: \href{https://leetcode.com/problems/two-sum/}{\texttt{LeetCode 1} --- \texttt{TwoSum}}
        \item Detect if Palindrome String: \href{https://leetcode.com/problems/valid-palindrome/description/}{\texttt{LeetCode 125} --- \texttt{ValidPalindrome}}
    \end{enumerate}
\end{alertblock}

\end{frame}

%%%%%%%%%%%%%%%%%%%%%%%%%%%%%%%%%%%%%%%%%%%%%%%%%%%%%%%%%%%%%%%%%%%%%%%%%%%%%%%


\begin{frame}[fragile]{Modules, Packages and Libraries}

\textbf{Module} --- A module is a file containing Python code that defines variables, functions, and classes. It can be imported and used in other Python files. \textbf{Example:} \texttt{math}, \texttt{os}, \texttt{random}

\vspace{.5cm}

\textbf{Package} --- A package is a directory that contains multiple Python modules, typically accompanied by an \_\_init\_\_.py file to indicate it's a package. \textbf{Example:} \texttt{numpy }, \texttt{pandas }, \texttt{matplotlib}

\vspace{.5cm}


\begin{block}{\textbf{Examples Imports}}
    \begin{itemize}
        \item \textbf{Module Import} --- \texttt{import urllib.request}
        \item \textbf{Function Import} --- \texttt{from urllib.request import urlopen}
        \item \textbf{Object Import} --- \texttt{import sys}
    \end{itemize}
\end{block}


\end{frame}

%%%%%%%%%%%%%%%%%%%%%%%%%%%%%%%%%%%%%%%%%%%%%%%%%%%%%%%%%%%%%%%%%%%%%%%%%%%%%%%


\begin{frame}[fragile]{Local Modules, PIP and PyPi}

\begin{block}{\textbf{Importing Local Modules}}
Local modules are Python files created and used within the same project

\begin{verbatim}
    def greet(name):
    return f"Hello, {name}!"
\end{verbatim}

\begin{verbatim}
    import my_module
    print(my_module.greet("Alice"))
\end{verbatim}

To create a local module, save a Python file \texttt{(.py)} and import its contents in another Python file using import
\end{block}

\vspace{.2cm}

\textbf{PyPI (Python Package Index)} --- PyPI is the official repository for Python packages, hosting thousands of installable Python packages/modules

\vspace{.2cm}

\textbf{PIP (Python Package Manager)} --- PIP is a package manager for Python, used to install, manage, and remove Python packages from PyPI

\end{frame}

%%%%%%%%%%%%%%%%%%%%%%%%%%%%%%%%%%%%%%%%%%%%%%%%%%%%%%%%%%%%%%%%%%%%%%%%%%%%%%%

\section{Interesting References}
\begin{frame}[fragile]{Interesting References}
\begin{exampleblock}{\textbf{Books}}
    
\begin{itemize}


\item \href{https://automatetheboringstuff.com/}{Automate the Boring Stuff with Python by Al Sweigart}

\item \href{http://greenteapress.com/thinkpython2/thinkpython2.pdf}{Think Python, 2nd Edition by Allen B. Downey}

\item \href{https://www.py4e.com/book.php}{Python for Everybody by Dr. Charles Severance}

\end{itemize}
\end{exampleblock}

\begin{exampleblock}{\textbf{Online Courses and Tutorials}}

\begin{itemize}

\item \href{https://www.codecademy.com/learn/learn-python-3}{Codecademy - Beginner Course}

\item \href{https://learnxinyminutes.com/docs/python/}{Learn X in Y Minutes - Python}

\item \href{https://ehmatthes.github.io/pcc_2e/cheat_sheets/cheat_sheets/}{Python Cheat Sheets by Eric Matthes}
\end{itemize}
\end{exampleblock}
\end{frame}


%%%%%%%%%%%%%%%%%%%%%%%%%%%%%%%%%%%%%%%%%%%%%%%%%%%%%%%%%%%%%%%%%%%%%%%%%%%%%%%

\begin{frame}[fragile]{The End}

\centering

\Huge Thank you!

\end{frame}


%------------------------------------------------

% \subsection{Columns}

% \begin{frame}
% 	\frametitle{Multiple Columns}
% 	\framesubtitle{Subtitle} % Optional subtitle
	
% 	\begin{columns}[c] % The "c" option specifies centered vertical alignment while the "t" option is used for top vertical alignment
% 		\begin{column}{0.45\textwidth} % Left column width
% 			\textbf{Heading}
% 			\begin{enumerate}
% 				\item Statement
% 				\item Explanation
% 				\item Example
% 			\end{enumerate}
% 		\end{column}
% 		\begin{column}{0.5\textwidth} % Right column width
% 			Lorem ipsum dolor sit amet, consectetur adipiscing elit. Integer lectus nisl, ultricies in feugiat rutrum, porttitor sit amet augue. Aliquam ut tortor mauris. Sed volutpat ante purus, quis accumsan dolor.
% 		\end{column}
% 	\end{columns}
% \end{frame}

\end{document} 