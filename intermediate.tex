%%%%%%%%%%%%%%%%%%%%%%%%%%%%%%%%%%%%%%%%%%%%%%%%%%%%%%%%%%%%%%%%%%%%%%%%%%%
% Beamer Presentation
% LaTeX Template
% Version 2.0 (March 8, 2022)
%
% This template originates from:
% https://www.LaTeXTemplates.com
%
% License:
% CC BY-NC-SA 4.0 (https://creativecommons.org/licenses/by-nc-sa/4.0/)
%
%%%%%%%%%%%%%%%%%%%%%%%%%%%%%%%%%%%%%%%%%%%%%%%%%%%%%%%%%%%%%%%%%%%%%%%%%%%
%%%%%%%%%%%%%%%%%%%%%%%%%%%%%%%%%%%%%%%%%%%%%%%%%%%%%%%%%%%%%%%%%%%%%%%%%%%
%
%%%%%%%%%%%%%%%%%%%%%%%%% MADE BY RICARDO CHIN %%%%%%%%%%%%%%%%%%%%%%%%%%%%
%
%%%%%%%%%%%%%%%%%%%%%%%%%%%%%%%%%%%%%%%%%%%%%%%%%%%%%%%%%%%%%%%%%%%%%%%%%%%
%%%%%%%%%%%%%%%%%%%%%%%%%%%%%%%%%%%%%%%%%%%%%%%%%%%%%%%%%%%% December 2023
%
% github.com/roaked 
% ricardochin.com
%                                                          git: MIT License
%----------------------------------------------------------------------------------------
%	PACKAGES AND OTHER DOCUMENT CONFIGURATIONS
%----------------------------------------------------------------------------------------

\documentclass[
	11pt, 
]{beamer}

\graphicspath{{Images/}{./}} 
\usepackage{graphicx}
\usepackage{booktabs} 
\usepackage{tikz}
\usetikzlibrary{shapes.arrows}
% Allows the use of \toprule, \midrule and \bottomrule for better rules in tables

\tikzset{
    myarrow/.style={
        draw,
        fill=orange,
        single arrow,
        minimum height=3.5ex,
        single arrow head extend=1ex
    }
}
\newcommand{\arrowup}{%
\tikz [baseline=-0.5ex]{\node [myarrow,rotate=90] {};}
}
\newcommand{\arrowdown}{%
\tikz [baseline=-1ex]{\node [myarrow,rotate=-90] {};}
}


%----------------------------------------------------------------------------------------
%	SELECT LAYOUT THEME
%----------------------------------------------------------------------------------------

\usetheme{Boadilla} %ok
%\usetheme{CambridgeUS} %red
%\usetheme{Goettingen} % ok
%\usetheme{Singapore} %good
%\usetheme{Szeged} %good

%----------------------------------------------------------------------------------------
%	SELECT COLOR THEME
%----------------------------------------------------------------------------------------


\usecolortheme{seahorse}


%----------------------------------------------------------------------------------------
%	SELECT FONT THEME & FONTS
%----------------------------------------------------------------------------------------

\usefonttheme{default}

%\usepackage{mathptmx} % Use the Times font for serif text
\usepackage{palatino} % Use the Palatino font for serif text
%\usepackage{helvet} % Use the Helvetica font for sans serif text
\usepackage[default]{opensans} % Use the Open Sans font for sans serif text
%\usepackage[default]{FiraSans} % Use the Fira Sans font for sans serif text
%\usepackage[default]{lato} % Use the Lato font for sans serif text

%----------------------------------------------------------------------------------------
%	SELECT INNER THEME
%----------------------------------------------------------------------------------------

%\useinnertheme{default}
\useinnertheme{circles}
%\useinnertheme{rectangles}
%\useinnertheme{rounded}
%\useinnertheme{inmargin}

%----------------------------------------------------------------------------------------
%	SELECT OUTER THEME
%----------------------------------------------------------------------------------------

%\useoutertheme{default}
%\useoutertheme{infolines}
%\useoutertheme{miniframes}
%\useoutertheme{smoothbars}
%\useoutertheme{sidebar}
%\useoutertheme{split}
%\useoutertheme{shadow}
%\useoutertheme{tree}
%\useoutertheme{smoothtree}

%----------------------------------------------------------------------------------
%	PRESENTATION INFORMATION
%----------------------------------------------------------------------------------

\title[Python: Early Fundamentals]{Intermediate Workshop to Python Programming}
\subtitle{Building the Foundation for Coding Success}
\author[Ricardo Chin]{Ricardo Chin}

\begin{document}
\begin{frame}
    \titlepage
    \begin{figure}
        \includegraphics[width=0.1\linewidth]{5848152fcef1014c0b5e4967}
    \end{figure}
    \frametitle{Workshop Details}
\end{frame}

%----------------------------------------------------------------------------------
%	TABLE OF CONTENTS SLIDE
%----------------------------------------------------------------------------------

\begin{frame}
	\frametitle{Presentation Overview}
	
	\tableofcontents
	%\tableofcontents[pausesections] % Output the table of contents (break sections up across separate slides)
\end{frame}


%%%%%%%%%%%%%%%%%%%%%%%%%%%%%%%%%%%%%%%%%%%%%%%%%%%%%%%%%%%%%%%%%%%%%%%%%%%%%%%

\section{Data Types} %change
\begin{frame}[fragile]{Data Types in Depth} %change

\begin{center}
--- Previously in \textsc{Python Programming} ---

\vspace{.25cm}
\arrowdown
\end{center}

\begin{itemize}
    \item \texttt{int}: Whole numbers without decimal points
    \item \texttt{float}: Numbers with decimal points
    \item \texttt{bool}: Represents the truth values \texttt{True} or \texttt{False}
    \item \texttt{NoneType} (\texttt{None}): Represents absence of a value (or null)
    \item \texttt{string}: Ordered sequence of characters 
\end{itemize}

\vspace{.2cm}

\textbf{Collections}
\begin{itemize}
    \item \texttt{list}: Ordered and mutable sequence of elements.
    \item \texttt{tuple}: Ordered and immutable sequence of elements.
    \item \texttt{dict}: Unordered collection of key-value pairs.
\end{itemize}

\end{frame}

%%%%%%%%%%%%%%%%%%%%%%%%%%%%%%%%%%%%%%%%%%%%%%%%%%%%%%%%%%%%%%%%%%%%%%%%%%%%%%%

\begin{frame}[fragile]{NoneType} %change

\textbf{NoneType}
\begin{itemize}
    \item \texttt{None} is a Singleton --- there is only ever a single instance of it inside a running Python program
    \item Multiple variables may refer to that same instance
\end{itemize}

\vspace{.2cm}

\textbf{Comparisons using Keyword \texttt{"is"}}

\vspace{.2cm}

Keyword \texttt{is} checks whether two names refer to the same object

\vspace{.4cm}

\begin{minipage}[t]{0.30\textwidth}
\begin{verbatim}
    a = [1, 2]
    b = a
    x = [1, 2]
    
    a == b # True
    a is b # True
    a == x # True
    a is x # False
\end{verbatim}
\end{minipage}
\hfill
\begin{minipage}[t]{0.50\textwidth}
As \texttt{None} is a singleton, we can check for it via \texttt{is None}

\begin{verbatim}
if a is None:
    print("a is None")
\end{verbatim}

\end{minipage}
\end{frame}

%%%%%%%%%%%%%%%%%%%%%%%%%%%%%%%%%%%%%%%%%%%%%%%%%%%%%%%%%%%%%%%%%%%%%%%%%%%%%%%

\begin{frame}[fragile]{BoolType} %change

\textbf{BoolType}
\vspace{.2cm}
\begin{itemize}
    \item The \texttt{bool} type is a built-in data type representing truth values.
    \item It has two possible values: \texttt{True} and \texttt{False}
\end{itemize}

\begin{minipage}[t]{0.30\textwidth}
\begin{verbatim}
a = True
if a:
    print('hello')
\end{verbatim}
\end{minipage}
\hfill
\begin{minipage}[t]{0.62\textwidth}
    \begin{verbatim}
x, y = 10, 20
is_greater = x > y   # False
if is_greater:
    print("x greater than y")
else:
    print("x not greater than y")
    \end{verbatim}
\end{minipage}

Booleans are a subset of integers (subclass of int) where \texttt{True} behaves as 1 and \texttt{False} as 0 in numerical contexts

\centering
\vspace{.4cm}
\texttt{False + True # 1}

\end{frame}

%%%%%%%%%%%%%%%%%%%%%%%%%%%%%%%%%%%%%%%%%%%%%%%%%%%%%%%%%%%%%%%%%%%%%%%%%%%%%%%

\begin{frame}[fragile]{Numbers} %change



\end{frame}



%%%%%%%%%%%%%%%%%%%%%%%%%%%%%%%%%%%%%%%%%%%%%%%%%%%%%%%%%%%%%%%%%%%%%%%%%%%%%%%

\section{Interesting References}
\begin{frame}[fragile]{Interesting References}
\begin{exampleblock}{\textbf{Books}}
    
\begin{itemize}


\item \href{https://automatetheboringstuff.com/}{Automate the Boring Stuff with Python by Al Sweigart}

\item \href{http://greenteapress.com/thinkpython2/thinkpython2.pdf}{Think Python, 2nd Edition by Allen B. Downey}

\item \href{https://www.py4e.com/book.php}{Python for Everybody by Dr. Charles Severance}

\end{itemize}
\end{exampleblock}

\begin{exampleblock}{\textbf{Online Courses and Tutorials}}

\begin{itemize}

\item \href{https://www.codecademy.com/learn/learn-python-3}{Codecademy - Beginner Course}

\item \href{https://learnxinyminutes.com/docs/python/}{Learn X in Y Minutes - Python}

\item \href{https://ehmatthes.github.io/pcc_2e/cheat_sheets/cheat_sheets/}{Python Cheat Sheets by Eric Matthes}
\end{itemize}
\end{exampleblock}
\end{frame}


%%%%%%%%%%%%%%%%%%%%%%%%%%%%%%%%%%%%%%%%%%%%%%%%%%%%%%%%%%%%%%%%%%%%%%%%%%%%%%%

\begin{frame}[fragile]{The End}

\centering

\Huge Thank you!

\end{frame}


%------------------------------------------------

% \subsection{Columns}

% \begin{frame}
% 	\frametitle{Multiple Columns}
% 	\framesubtitle{Subtitle} % Optional subtitle
	
% 	\begin{columns}[c] % The "c" option specifies centered vertical alignment while the "t" option is used for top vertical alignment
% 		\begin{column}{0.45\textwidth} % Left column width
% 			\textbf{Heading}
% 			\begin{enumerate}
% 				\item Statement
% 				\item Explanation
% 				\item Example
% 			\end{enumerate}
% 		\end{column}
% 		\begin{column}{0.5\textwidth} % Right column width
% 			Lorem ipsum dolor sit amet, consectetur adipiscing elit. Integer lectus nisl, ultricies in feugiat rutrum, porttitor sit amet augue. Aliquam ut tortor mauris. Sed volutpat ante purus, quis accumsan dolor.
% 		\end{column}
% 	\end{columns}
% \end{frame}

\end{document} 